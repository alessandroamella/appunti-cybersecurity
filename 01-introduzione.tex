\input{preambolo_comune}

% --- Titolo ---
\title{Introduzione}
\author{Basato sulle slide della Prof.ssa Jocelyne Elias}
\date{\today}

\begin{document}

\maketitle
\tableofcontents
\newpage

\section{Informazioni Generali sul Corso}

\subsection{Docente e Contatti}
\begin{itemize}
    \item \textbf{Docente:} Jocelyne Elias
    \item \textbf{Email:} \href{mailto:jocelyne.elias@unibo.it}{\texttt{jocelyne.elias@unibo.it}}
    \item \textbf{Website Docente:} \url{https://www.unibo.it/sitoweb/jocelyne.elias/}
\end{itemize}

\subsection{Orario Lezioni e Ricevimento}
\begin{itemize}
    \item \textbf{Orario Lezioni:}
    \begin{itemize}
        \item \textbf{Giovedì:} 09:00 – 12:00, AULA CREMONA
        \item \textbf{Venerdì:} 11:00 – 14:00, AULA L.TONELLI
        \item Per orari aggiornati (anno 2024, codice insegnamento 455457):\\ \url{https://www.unibo.it/it/studiare/dottorati-master-specializzazioni-e-altra-formazione/insegnamenti/insegnamento/2024/455457/orariolezioni}
    \end{itemize}
    \item \textbf{Ricevimento:} Da concordare via mail con la docente.
\end{itemize}

\subsection{Piattaforma Didattica}
\begin{itemize}
    \item Si utilizzerà la piattaforma \textbf{Virtuale}.
    \item Sarà costantemente aggiornata con:
    \begin{itemize}
        \item Informazioni generali
        \item \textbf{Materiale didattico (slides)} - Risorsa principale.
        \item Annunci
    \end{itemize}
\end{itemize}

\section{Materiale Didattico}
\begin{itemize}
    \item \textbf{Slide:} Caricate su Virtuale (fondamentali per lo studio).
    \item \textbf{Testi Consigliati (per approfondimento):}
    \begin{enumerate}
        \item William Stallings, Lawrie Brown, \textit{Computer Security Principles and Practice (Fifth Edition)}
        \item William Stallings, \textit{Cryptography and Network Security: Principles and Practice (Eighth Edition)}
        \item Douglas R. Stinson, Maura B. Paterson, \textit{Cryptography Theory and Practice}
        \item Dan Boneh, Victor Shoup, \textit{A Graduate Course in Applied Cryptography} (approccio matematico)
        \item Bruce Schneier, \textit{Applied Cryptography: Protocols, Algorithms, and Source Code in C.}
    \end{enumerate}
\end{itemize}

\section{Modalità d'Esame}
\begin{itemize}
    \item \textbf{Prova scritta} (NON ci sono orali).
    \item La prova scritta include una \textbf{domanda o un esercizio} (massimo 6 punti) sulla \textbf{parte dei laboratori}.
    \item \textbf{Date Esami:} Consultare il sito del Dipartimento.
    \begin{itemize}
        \item \textbf{Sessione Estiva (giugno-luglio):} Tre appelli.
        \item \textbf{Sessione Autunnale (settembre):} Un appello.
        \item \textbf{Sessione Invernale (gennaio/febbraio 2026):} Due appelli.
    \end{itemize}
\end{itemize}

\section{Roadmap del Corso (Argomenti Trattati)}
\begin{enumerate}
    \item Introduzione alla Cybersecurity
    \item Crittografia
    \begin{itemize}
        \item Chiave Simmetrica
        \item Chiave Asimmetrica
    \end{itemize}
    \item Sicurezza dei Sistemi e Permessi
    \item Password e Autenticazione
    \item Sicurezza Internet, Comunicazione Anonima, Privacy dei Dati
    \item Sicurezza delle Reti Wireless
    \item Sicurezza del Software: Buffer Overflows
    \item Sicurezza Web (indicato con "?")
    \item \dots (altri argomenti)
\end{enumerate}

\section{Concetti Fondamentali di Cybersecurity}

\subsection{Cos'è la Cybersecurity?}
La Cybersecurity si riferisce a qualsiasi \textbf{tecnologia, misura o pratica} volta a \textbf{prevenire gli attacchi informatici (cyberattacks)} o a mitigarne l'impatto.
L'obiettivo è proteggere sistemi, applicazioni, dispositivi, dati sensibili e asset finanziari.

Secondo Gartner, la Cybersecurity è la pratica di impiegare \textbf{persone, policy, processi e tecnologie} per proteggere le organizzazioni da attacchi digitali.
\begin{itemize}
    \item \textit{Esempio Pratico:} Un'azienda che implementa la cybersecurity non si limita a installare un firewall (tecnologia). Forma i propri dipendenti (persone) su come riconoscere email di phishing (policy e processi), stabilisce regole per l'accesso ai dati (policy) e ha procedure definite (processi) per rispondere a un incidente di sicurezza.
\end{itemize}

\subsection{Cos'è un Cyberattack (Attacco Informatico)?}
Un cyberattack è qualsiasi \textbf{sforzo intenzionale} per rubare, esporre, alterare, disabilitare o distruggere dati, applicazioni o altri asset, attraverso un \textbf{accesso non autorizzato}.
\begin{itemize}
    \item \textit{Esempio Pratico:} Un hacker che ottiene l'accesso al database di un sito e-commerce e ruba i numeri delle carte di credito dei clienti.
\end{itemize}

\subsection{Tipi Comuni di Attacchi Informatici}
\begin{enumerate}
    \item \textbf{Phishing e Ingegneria Sociale:} Ingannare utenti per ottenere credenziali o compiere azioni dannose.
        \begin{itemize}
            \item \textit{Esempio:} Email fasulla della banca che chiede di aggiornare i dati su un sito clone.
        \end{itemize}
    \item \textbf{Rischi dei Servizi Esposti su Internet (inclusi Cloud Services):} Mancata messa in sicurezza di servizi cloud o esposti (es. errori di configurazione).
        \begin{itemize}
            \item \textit{Esempio:} Database su cloud accessibile pubblicamente senza password.
        \end{itemize}
    \item \textbf{Compromissione di Account legata alle Password:} Uso di password comuni, riutilizzate o deboli.
        \begin{itemize}
            \item \textit{Esempio:} Attaccante prova "password123"; credential stuffing.
        \end{itemize}
    \item \textbf{Uso Improprio di Informazioni (Misuse of Information):} Utenti autorizzati che diffondono o usano impropriamente dati.
        \begin{itemize}
            \item \textit{Esempio:} Dipendente scontento copia l'elenco clienti; invio errato di email con dati sensibili.
        \end{itemize}
    \item \textbf{Attacchi di Rete e Man-in-the-Middle (MitM):} Intercettazione, reindirizzamento o interruzione del traffico di rete non sicuro.
        \begin{itemize}
            \item \textit{Esempio:} Su Wi-Fi pubblico non sicuro, un attaccante intercetta traffico non cifrato.
        \end{itemize}
    \item \textbf{Attacchi alla Supply Chain:} Compromissione di partner, fornitori o codice di terze parti.
        \begin{itemize}
            \item \textit{Esempio (Dependency Confusion):} Attaccante pubblica libreria malevola con lo stesso nome di una interna, inducendo il sistema di build a scaricarla.
        \end{itemize}
    \item \textbf{Attacchi Denial-of-Service (DoS) e Distributed DoS (DDoS):} Sovraccarico di sistemi con traffico massiccio.
        \begin{itemize}
            \item \textit{Esempio:} Sito web bombardato da richieste da una botnet, rendendolo irraggiungibile.
        \end{itemize}
    \item \textbf{Ransomware (\$):} Software malevolo che cifra dati o blocca sistemi chiedendo un riscatto.
        \begin{itemize}
            \item \textit{Esempio:} Utente apre allegato infetto, ransomware cifra i file aziendali, compare richiesta di riscatto.
        \end{itemize}
\end{enumerate}

\section{Breve Contesto Storico}

\subsection{ARPANET}
\begin{itemize}
    \item \textbf{1969:} ARPANET (precursore di Internet) va online.
    \item \textbf{1973:} Robert Metcalfe avverte che ARPANET è insicura.
    \item \textbf{1983:} Fred Cohen conia il termine "computer virus".
    \item \textbf{1983:} ARPANET adotta il protocollo TCP/IP.
\end{itemize}

\subsection{Il Primo Worm: Morris Worm (1988)}
\begin{itemize}
    \item \textbf{Robert Morris} rilascia inavvertitamente il primo worm.
    \item \textbf{Funzionamento:} Scopriva host, sfruttava vulnerabilità in \texttt{sendmail}, si copiava.
    \item \textbf{Intento:} Esperimento di ricerca.
    \item \textbf{Problema:} Bug causò replicazione fuori controllo.
    \item \textbf{Impatto:} Crash del 10\% dei computer su ARPANET.
    \item \textbf{Conseguenze Legali:} Condanna secondo CFAA, libertà vigilata, multa \$10k.
    \item \textbf{Significato Tecnico:} Primo uso documentato di exploit di \textbf{buffer overflow}.
    \begin{itemize}
        \item \textit{Buffer Overflow:} Inserire più dati di quanti un buffer possa contenere, sovrascrivendo memoria adiacente, potenzialmente per eseguire codice malevolo.
    \end{itemize}
\end{itemize}

\section{Definizioni Chiave Aggiuntive}

\subsection{Cyber Systems (Sistemi Cyber)}
La cybersecurity riguarda i \textbf{Cyber Systems}, che integrano:
\begin{enumerate}
    \item Computer
    \item Comunicazioni (reti)
    \item Persone (utenti e operatori)
\end{enumerate}
\begin{itemize}
    \item \textit{Esempio Pratico:} Un sistema SCADA per una centrale elettrica (computer, reti, operatori).
\end{itemize}

\subsection{Vulnerability, Attack, Threat}
\begin{itemize}
    \item \textbf{Vulnerability (Vulnerabilità):} Debolezza che può essere sfruttata.
        \begin{itemize}
            \item \textit{Esempio:} Software non aggiornato; password "123456".
        \end{itemize}
    \item \textbf{Attack (Attacco):} Metodo per sfruttare una vulnerabilità.
        \begin{itemize}
            \item \textit{Esempio:} Usare un exploit; provare password deboli.
        \end{itemize}
    \item \textbf{Threat (Minaccia):} Avversario motivato e capace che attacca.
        \begin{itemize}
            \item \textit{Esempio:} Gruppo criminale che lancia phishing per rubare dati.
        \end{itemize}
\end{itemize}

\subsection{Strategie di Difesa}
\begin{enumerate}
    \item \textbf{Identificare e correggere ogni vulnerabilità} (es. patch, password complesse).
    \item \textbf{Identificare gli attacchi ed eliminare le vulnerabilità} che sfruttano (es. monitoraggio log, analisi).
\end{enumerate}

\section{Zero-Day}

\subsection{Definizioni}
\begin{itemize}
    \item \textbf{Zero-Day Vulnerability:} Vulnerabilità \textbf{non nota} a chi dovrebbe mitigarla.
        \begin{itemize}
            \item \textit{Esempio:} Hacker scopre nuovo bug in Windows, Microsoft non lo sa ancora.
        \end{itemize}
    \item \textbf{Window of Opportunity (Finestra di Opportunità):} Il \textbf{tempo} tra l'attivazione di un exploit e l'applicazione della patch. Periodo in cui il sistema è vulnerabile.
    \item \textbf{Zero-Day Attack:} Attacco che avviene \textbf{durante la finestra di opportunità}, sfruttando una vulnerabilità zero-day.
        \begin{itemize}
            \item \textit{Esempio:} Exploit venduto nel dark web e usato prima della patch.
        \end{itemize}
\end{itemize}

\subsection{Andamento e Fonti delle Vulnerabilità Zero-Day}
\begin{itemize}
    \item \textbf{Andamento Finestra di Opportunità:}
    \begin{itemize}
        \item \textbf{2005:} Media 54 giorni.
        \item \textbf{2014:} Cresciuta a quasi 12 mesi.
    \end{itemize}
    \item \textbf{Fonti:} Spesso le grandi aziende tecnologiche (Big Tech) a causa della complessità e diffusione dei loro OS (Windows, Android, iOS, MacOS).
    \item \textbf{Grafico - Zero Days of 2022 by Vendor (Placeholder):}
    \begin{figure}[H]
        \centering
        \includegraphics[width=0.7\textwidth]{images/zero_days_by_vendor_2022.png}
        \caption{Zero Days of 2022 by Vendor (Fonte: Slide 20, da generare con Python).}
        \label{fig:zero_days_vendor}
    \end{figure}
    \item \textbf{Grafico - Zero-Day Vulnerabilities by Year (Placeholder):}
    \begin{figure}[H]
        \centering
        \includegraphics[width=0.9\textwidth]{images/zero_day_vulnerabilities_by_year.png}
        \caption{Zero-Day Vulnerabilities by Year (Fonte: Slide 21, Google Project Zero, da generare con Python).}
        \label{fig:zero_days_year}
    \end{figure}
    \textit{Nota: I grafici qui sopra sono placeholder. Lo script Python per generarli è fornito separatamente.}
\end{itemize}

\subsection{Esempi Noti di Exploit Zero-Day}
\begin{itemize}
    \item \textbf{Code Red Worm (2001):} Buffer overflow in Microsoft IIS.
    \item \textbf{Heartbleed (2014):} Bug in OpenSSL, esponeva memoria server.
        \begin{itemize}
            \item \textit{Esempio Heartbleed:} Attaccante otteneva dati (credenziali, chiavi SSL) dalla memoria di server vulnerabili.
        \end{itemize}
    \item \textbf{Shellshock (2014):} Vulnerabilità in Bash Unix per RCE.
    \item \textbf{Petya/NotPetya (2017):} Ransomware, sfruttava EternalBlue.
    \item \textbf{WannaCry (2017):} Ransomware, sfruttava EternalBlue.
\end{itemize}

\section{Cybersecurity in Ambiti Specifici}
Applicazioni in:
\begin{itemize}
    \item Applicazioni IoT (Internet of Things)
    \item Applicazioni Healthcare (Sanità)
\end{itemize}

\subsection{Applicazioni IoT (Internet of Things)}

\subsubsection{Architettura a Livelli per la Sicurezza IoT}
Un modello comune (Lu \& Xu, IEEE IoT Journal 2019) per analizzare la sicurezza IoT è un'architettura a 4 livelli.

\begin{table}[H]
\centering
\caption{Architettura a 4 Livelli per la Sicurezza IoT (Adattato da Lu \& Xu, 2019)}
\label{tab:iot_architecture}
\begin{tabular}{|p{0.15\textwidth}|p{0.4\textwidth}|p{0.35\textwidth}|}
\hline
\rowcolor{bg_custom}
\textcolor{black}{\textbf{Livello}} & \textcolor{black}{\textbf{Descrizione e Focus Attacco}} & \textcolor{black}{\textbf{Tipi di Attacco}} \\
\hline
\textbf{Sensing} & Oggetti fisici e sensori. \newline \textbf{Focus:} confidenzialità. &
    \begin{itemize}[nosep, leftmargin=*, itemsep=-2pt, before=\vspace{-0.5\baselineskip}, after=\vspace{-\baselineskip}]
        \item Replay Attacks
        \item Timing Attacks
        \item Node Capture
        \item Malicious Data Attacks
        \item SCA (Side Channel Attack)
    \end{itemize} \\
\hline
\textbf{Networking} & Trasmissione dati. \newline \textbf{Focus:} confidenzialità, privacy, compatibilità. &
    \begin{itemize}[nosep, leftmargin=*, itemsep=-2pt, before=\vspace{-0.5\baselineskip}, after=\vspace{-\baselineskip}]
        \item Spoofed/altered/ replayed routing
        \item Sybil attacks
        \item Wormholes
    \end{itemize} \\
\hline
\textbf{Middleware} & Consegna dati. \newline \textbf{Focus:} autenticità, integrità, confidenzialità. &
    \begin{itemize}[nosep, leftmargin=*, itemsep=-2pt, before=\vspace{-0.5\baselineskip}, after=\vspace{-\baselineskip}]
        \item Malicious Insider
        \item Attacchi infrastruttura
        \item Minacce terze parti
        \item Minacce virtualizzazione
    \end{itemize} \\
\hline
\textbf{Application} & Fornitura servizi. \newline \textbf{Focus:} privacy dati/identità, autenticazione. &
    \begin{itemize}[nosep, leftmargin=*, itemsep=-2pt, before=\vspace{-0.5\baselineskip}, after=\vspace{-\baselineskip}]
        \item Phishing Attack
        \item Virus, Worms, Trojan
        \item Spyware, Malicious Scripts
        \item Unauthorized Access
    \end{itemize} \\
\hline
\end{tabular}
\end{table}

\subsubsection{Componenti dell'Edge Layer in Architetture IoT}
Diagramma che illustra i flussi tipici nell'edge layer.
\begin{figure}[H]
\centering
\begin{tikzpicture}[node distance=2.5cm and 1.5cm, auto]
    % Nodi
    \node[cloud_white, text width=4em] (cloud) {Cloud Services};
    \node[sensor_white, below left=0.8cm and 0.3cm of cloud, label={[label_white]below:IoT Nodes}] (iot_node1) {N1};
    \node[sensor_white, below right=0.8cm and 0.3cm of cloud, label={[label_white]below:IoT Nodes}] (iot_node2) {N2};
    \node[block_white, text width=4em, below=1.5cm of cloud] (gateway) {IoT Gateway};
    \node[block_white, text width=5em, below left=0.8cm and 0.3cm of gateway] (mobile) {Mobile Devices};
    \node[sensor_white, right=1.5cm of gateway, label={[label_white]right:Sensors}] (sensor1) {S1};
    \node[sensor_white, above right=0.2cm and 0.1cm of sensor1, label={[label_white]right:Sensors}] (sensor2) {S2};
    \node[sensor_white, below right=0.2cm and 0.1cm of sensor1, label={[label_white]right:Sensors}] (sensor3) {S3};

    % Connessioni
    \path[line_white] (iot_node1.east) -- (gateway.west);
    \path[line_white] (iot_node2.east) -- (gateway.west);
    \path[line_white] (gateway.north) -- (cloud.south);
    \path[line_white] (mobile.north) -- (gateway.south west);
    \path[line_white] (gateway.east) -- (sensor1.west);
    \path[line_white] (gateway.east) -- (sensor2.west);
    \path[line_white] (gateway.east) -- (sensor3.west);
\end{tikzpicture}
\caption{Componenti dell'Edge Layer in Architetture IoT (Adattato da Slide 25).}
\label{fig:iot_edge_layer}
\end{figure}

\subsubsection{Tassonomia Applicazioni e Sfide di Sicurezza (Descrizione)}
Le slide 26 e 27 presentano tassonomie complesse per le applicazioni IoT e per i requisiti/sfide di sicurezza, suddivise per layer architetturale (Edge, Network, Application).
\begin{itemize}
    \item \textbf{Requisiti di Sicurezza per Layer:}
    \begin{itemize}
        \item \textit{Edge:} Crittografia, Threat Hunting, Accesso Limitato, Autenticazione.
        \item \textit{Network:} Traffic Shaping, Rilevamento Anomalie, Monitoraggio Traffico.
        \item \textit{Application:} Analisi Forense, API Sicure, Verifica Applicazioni.
    \end{itemize}
    \item \textbf{Sfide (Esempi di Attacchi per Layer):}
    \begin{itemize}
        \item \textit{Edge:} Deprivation Attack, Interfacce Insicure, Sybil/Spoofing, Jamming.
        \item \textit{Network:} Violazione Privacy Cloud, Session Hijacking, Sinkhole/Wormhole.
        \item \textit{Application:} Replay Attack, Middleware Insicuro, Software/OS Insicuri.
    \end{itemize}
\end{itemize}

\subsection{Sicurezza e Privacy in Healthcare 4.0}

\subsubsection{Architettura Generale e Minacce (Descrizione)}
La slide 28 mostra un ecosistema Healthcare 4.0 complesso con sensori indossabili, dispositivi medici, cloud, fog computing, ospedali. Evidenzia attacchi a vari livelli e menziona la Blockchain per la sicurezza dei dati.

\subsubsection{Scambio Dati Mobile tra Operatori Sanitari}
Illustrazione del flusso di dati paziente-caregiver.
\begin{figure}[H]
\centering
\begin{tikzpicture}[node distance=1.5cm and 1cm]
    % Nodi
    \node[block_white, text width=5em] (patient) {Paziente (Sensori IoT)};
    \node[block_white, text width=7em, right=of patient] (patient_mobile) {Stazione Mobile Paziente};
    \node[cloud_white, text width=3em, right=of patient_mobile] (network) {3G/4G/5G};
    \node[block_white, text width=7em, right=of network] (doctor_mobile) {Stazione Mobile Dottore};
    \node[block_white, text width=4em, below right=0.5cm and -1cm of doctor_mobile] (hospital) {Ospedale};
    \node[block_white, text width=4em, below left=0.5cm and -1cm of doctor_mobile] (pharmacy) {Farmacia};

    % Connessioni
    \path[line_white] (patient) -- node[above, label_white] {Dati} (patient_mobile);
    \path[line_white] (patient_mobile) -- (network);
    \path[line_white] (network) -- (doctor_mobile);
    \path[line_white] (doctor_mobile.south) |- ($(doctor_mobile.south)!0.5!(hospital.north west)$) -- (hospital.north west);
    \path[line_white] (doctor_mobile.south) |- ($(doctor_mobile.south)!0.5!(pharmacy.north east)$) -- (pharmacy.north east);

    % Etichette Location send
    \node[label_white, rotate=30] at ($(patient_mobile)!0.5!(network)$) [above=0.1cm] {\textit{Location send}};
    \node[label_white, rotate=-30] at ($(network)!0.5!(doctor_mobile)$) [above=0.1cm] {\textit{Location send}};

\end{tikzpicture}
\caption{Scambio Dati Mobile in Healthcare 4.0 (Adattato da Slide 29).}
\label{fig:healthcare_mobile_data}
\end{figure}

\subsubsection{Possibili Attacchi su Diversi Livelli in Healthcare}
La slide 30 mostra i livelli (Fisico, Comunicazione, Applicazione) e i relativi attacchi.
\begin{figure}[H]
\centering
\begin{tikzpicture}[node distance=1cm and 0.5cm]
    % Layers
    \node[block_wide_white, minimum height=10em] (physical_layer_box) {};
    \node[label_white, anchor=south] at (physical_layer_box.north) {Livello Fisico (Sensori)};
     \node[below=0.2cm of physical_layer_box.north, text width=10em, align=center] (physical_content) {
        \textit{Attacchi:}\newline Eavesdropping, Hardware Tempering, Physical Attack, Malware, DoS
    };


    \node[block_wide_white, minimum height=12em, above=0.5cm of physical_layer_box] (comm_layer_box) {};
    \node[label_white, anchor=south] at (comm_layer_box.north) {Livello Comunicazione};
    \node[below=0.2cm of comm_layer_box.north, text width=10em, align=center] (comm_content) {
        \textit{Attacchi:}\newline Sybil, Eavesdropping, Replay, MiM, Privacy Threat, Data Corruption, Malware, DDoS, Identity Theft, Routing, Spoofing, Passive Monitoring
    };

    \node[block_wide_white, minimum height=8em, above=0.5cm of comm_layer_box] (app_layer_box) {};
    \node[label_white, anchor=south] at (app_layer_box.north) {Livello Applicazione};
     \node[below=0.2cm of app_layer_box.north, text width=10em, align=center] (app_content) {
        \textit{Attacchi:}\newline SQL Injection, Phishing, Privacy Threat, Data Corruption, Malware, DDoS
    };


    % Decorazioni (frecce attacchi)
    \foreach \layeranchor/\textanchor/\attacktext [count=\i] in {
        app_layer_box.west/left/{SQLi, Phishing},
        comm_layer_box.west/left/{Sybil, Eavesdrop, Replay, MiM},
        physical_layer_box.west/left/{Eavesdrop, HW Temper}
    }{
        \node[label_white, anchor=east, xshift=-0.5cm] at (\layeranchor) {\attacktext};
        \draw[line_white, thick] ($( \layeranchor)+(-0.4cm,0)$) -- (\layeranchor);
    }
    \foreach \layeranchor/\textanchor/\attacktext [count=\i] in {
        app_layer_box.east/right/{Priv. Threat, Data Corr., Malware, DDoS},
        comm_layer_box.east/right/{ID Theft, Routing, Spoof, DDoS, Pass. Mon.},
        physical_layer_box.east/right/{Phys. Atk, Malware, DoS}
    }{
        \node[label_white, anchor=west, xshift=0.5cm, text width=5em, align=left] at (\layeranchor) {\attacktext};
        \draw[line_white, thick] (\layeranchor) -- ($( \layeranchor)+(0.4cm,0)$);
    }

\end{tikzpicture}
\caption{Possibili Attacchi per Livello in Healthcare 4.0 (Adattato da Slide 30).}
\label{fig:healthcare_attacks_layer}
\end{figure}

\end{document}