\include{preambolo_comune}

\title{Dimostrazione Perfect Secrecy dello Shift Cipher}
\author{Alessandro Amella, Gemini e Claude}
\date{\today}

\begin{document}

\maketitle

\section{Cos'è il Shift Cipher}

Il \textbf{Shift Cipher} (o Cifrario di Cesare) è uno dei crittosistemi più semplici. L'idea è di "spostare" ogni lettera dell'alfabeto di un numero fisso di posizioni.

\subsection{Esempio pratico}
Se uso chiave $k = 3$:
\begin{itemize}
    \item A (posizione 0) $\to$ D (posizione 3)
    \item B (posizione 1) $\to$ E (posizione 4)  
    \item C (posizione 2) $\to$ F (posizione 5)
    \item ...
    \item X (posizione 23) $\to$ A (posizione 0, perché $(23+3) \bmod 26 = 0$)
\end{itemize}

Quindi "HELLO" diventa "KHOOR".

\section{Notazione Matematica}

\subsection{Gli insiemi P, C, K}
\begin{itemize}
    \item $P$ = \textbf{Plaintext space} = insieme di tutti i possibili messaggi in chiaro
    \item $C$ = \textbf{Ciphertext space} = insieme di tutti i possibili messaggi cifrati  
    \item $K$ = \textbf{Key space} = insieme di tutte le possibili chiavi
\end{itemize}

Per il Shift Cipher:
\[P = C = K = \mathbb{Z}_{26} = \{0, 1, 2, \ldots, 25\}\]

Questo significa:
\begin{itemize}
    \item Ogni lettera (A=0, B=1, ..., Z=25) può essere plaintext
    \item Ogni lettera può essere ciphertext
    \item Ogni numero da 0 a 25 può essere una chiave (shift)
\end{itemize}

\subsection{Le funzioni di cifratura e decifratura}

\textbf{Cifratura}: $e_k(x) = (x + k) \bmod 26$
\begin{itemize}
    \item $x$ = lettera del plaintext (come numero 0-25)
    \item $k$ = chiave (shift da applicare)  
    \item Sommiamo $k$ a $x$ e prendiamo il resto della divisione per 26
\end{itemize}

\textbf{Decifratura}: $d_k(y) = (y - k) \bmod 26$  
\begin{itemize}
    \item $y$ = lettera del ciphertext
    \item Sottraiamo $k$ per "annullare" lo shift
\end{itemize}

\subsection{Esempio numerico}
Cifriamo "CAT" con chiave $k = 5$:
\begin{align}
C &= 2 \quad \to \quad e_5(2) = (2 + 5) \bmod 26 = 7 = H \\
A &= 0 \quad \to \quad e_5(0) = (0 + 5) \bmod 26 = 5 = F \\  
T &= 19 \quad \to \quad e_5(19) = (19 + 5) \bmod 26 = 24 = Y
\end{align}

Quindi "CAT" $\to$ "HFY".

Per decifrare "HFY":
\begin{align}
H &= 7 \quad \to \quad d_5(7) = (7 - 5) \bmod 26 = 2 = C \\
F &= 5 \quad \to \quad d_5(5) = (5 - 5) \bmod 26 = 0 = A \\
Y &= 24 \quad \to \quad d_5(24) = (24 - 5) \bmod 26 = 19 = T  
\end{align}

\section{Perfect Secrecy: Definizione}

\textbf{Definizione}: Un crittosistema ha Perfect Secrecy se osservare il ciphertext non dà \textit{nessuna informazione} sul plaintext originale.

Matematicamente:
\[\Pr[X = x \mid Y = y] = \Pr[X = x]\]

Dove:
\begin{itemize}
    \item $X$ = variabile aleatoria del plaintext
    \item $Y$ = variabile aleatoria del ciphertext  
    \item $\Pr[X = x \mid Y = y]$ = probabilità che il plaintext sia $x$, sapendo che il ciphertext è $y$
    \item $\Pr[X = x]$ = probabilità a priori che il plaintext sia $x$ (senza vedere il ciphertext)
\end{itemize}

\textbf{Intuizione}: Se queste due probabilità sono uguali, significa che vedere il ciphertext $y$ non cambia le nostre credenze sul plaintext!

\section{Dimostrazione: Il Shift Cipher ha Perfect Secrecy}

\subsection{Assunzioni}
\begin{itemize}
    \item Le chiavi sono scelte uniformemente: $\Pr[K = k] = \frac{1}{26}$ per ogni $k$
    \item Il plaintext può essere qualsiasi lettera con qualche probabilità $\Pr[X = x]$
\end{itemize}

\subsection{Passo 1: Relazione chiave-plaintext-ciphertext}
Per ogni coppia plaintext-ciphertext $(x, y)$, esiste \textbf{esattamente una chiave} che li collega:
\[k = (y - x) \bmod 26\]

\textbf{Esempio}: Se plaintext è C (=2) e ciphertext è H (=7), allora:
\[k = (7 - 2) \bmod 26 = 5\]

\subsection{Passo 2: Probabilità congiunta}
\begin{align}
\Pr[X = x, Y = y] &= \Pr[X = x] \cdot \Pr[K = (y-x) \bmod 26] \\
&= \Pr[X = x] \cdot \frac{1}{26}
\end{align}

\subsection{Passo 3: Probabilità marginale del ciphertext}
Un ciphertext $y$ può derivare da \textit{qualsiasi} plaintext $x$, basta usare la chiave giusta:
\begin{align}
\Pr[Y = y] &= \sum_{x=0}^{25} \Pr[X = x, Y = y] \\
&= \sum_{x=0}^{25} \Pr[X = x] \cdot \frac{1}{26} \\
&= \frac{1}{26} \sum_{x=0}^{25} \Pr[X = x] \\
&= \frac{1}{26} \cdot 1 = \frac{1}{26}
\end{align}

\textbf{Insight importante}: Ogni ciphertext ha la stessa probabilità $\frac{1}{26}$, indipendentemente dalla distribuzione del plaintext!

\subsection{Passo 4: Probabilità condizionale}
\begin{align}
\Pr[X = x \mid Y = y] &= \frac{\Pr[X = x, Y = y]}{\Pr[Y = y]} \\
&= \frac{\Pr[X = x] \cdot \frac{1}{26}}{\frac{1}{26}} \\
&= \Pr[X = x]
\end{align}

\section{Conclusione e Interpretazione}

\textbf{Risultato}: Il Shift Cipher ha Perfect Secrecy quando le chiavi sono uniformemente distribuite.

\textbf{Perché funziona?}
\begin{itemize}
    \item Ogni ciphertext può provenire da qualsiasi plaintext con uguale probabilità
    \item La "maschera" della chiave casuale nasconde completamente l'informazione
    \item Un attaccante che vede solo il ciphertext non può distinguere tra plaintext diversi
\end{itemize}

\textbf{Esempio pratico}: Se intercetti il ciphertext "H", potrebbe essere:
\begin{itemize}
    \item "A" cifrato con chiave 7
    \item "B" cifrato con chiave 6  
    \item "C" cifrato con chiave 5
    \item ... (tutte le 26 possibilità sono ugualmente probabili!)
\end{itemize}

Senza informazioni aggiuntive, è impossibile determinare il plaintext originale. \qed

\end{document}