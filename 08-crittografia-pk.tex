\input{preambolo_comune}

\title{Crittografia a Chiave Pubblica/Asimmetrica}
\author{Basato sulle slide della Prof.ssa Jocelyne Elias}
\date{\today}

\begin{document}

\maketitle
\tableofcontents
\newpage

\section{Principi Generali dei Crittosistemi a Chiave Pubblica}

\subsection{Problemi con la Crittografia Simmetrica}
La crittografia simmetrica (es. 3DES, AES) utilizza la \textbf{stessa chiave segreta} sia per cifrare che per decifrare.

\begin{figure}[H]
    \centering
    \begin{tikzpicture}[
        box/.style={draw, minimum width=2cm, minimum height=1cm},
        arrow/.style={->, thick},
        key/.style={draw, circle, minimum size=0.8cm}
    ]
        % Alice side
        \node[box] (alice) at (0,0) {Alice};
        \node[box] (enc) at (3,0) {Cifratura};
        
        % Network
        \node[cloud, draw, minimum width=2.5cm] (net) at (6,0) {Rete};
        
        % Bob side
        \node[box] (dec) at (9,0) {Decifratura};
        \node[box] (bob) at (12,0) {Bob};
        
        % Keys
        \node[key] (key1) at (3,-2) {K};
        \node[key] (key2) at (9,-2) {K};
        
        % Arrows
        \draw[arrow] (alice) -- (enc);
        \draw[arrow] (enc) -- (net);
        \draw[arrow] (net) -- (dec);
        \draw[arrow] (dec) -- (bob);
        
        % Key connections
        \draw[arrow, dashed, red] (key1) -- (enc);
        \draw[arrow, dashed, red] (key2) -- (dec);
        \draw[arrow, dashed, red] (key1) to[bend right=20] node[below] {Scambio sicuro necessario} (key2);
    \end{tikzpicture}
    \caption{Schema semplificato della crittografia simmetrica. La stessa chiave K deve essere condivisa in modo sicuro tra mittente e destinatario.}
    \label{fig:simmetrica}
\end{figure}

I principali problemi della crittografia simmetrica sono:

\begin{itemize}
    \item \textbf{Scambio della Chiave:} Come possono Alice e Bob scambiarsi la chiave segreta in modo sicuro, specialmente se non si sono mai incontrati? Questo è il problema della \textit{distribuzione della chiave}.
    \item \textbf{Gestione delle Chiavi:} Per comunicare con $n$ persone diverse in modo sicuro, servono $n(n-1)/2$ chiavi diverse. Ad esempio, per 100 persone servono 4950 chiavi!
\end{itemize}

\subsection{Motivazione della Crittografia Asimmetrica}
\begin{itemize}
    \item \textbf{Domanda Fondamentale:} È possibile per Alice e Bob, che non hanno mai condiviso una chiave segreta, comunicare in modo sicuro?
    \item \textbf{Soluzione:} La crittografia asimmetrica.
\end{itemize}

\subsection{Funzionamento della Crittografia Asimmetrica}
Ogni utente genera una \textbf{coppia di chiavi} matematicamente correlate:
\begin{itemize}
    \item Una \textbf{Chiave Pubblica (\texttt{pk})}: Può essere distribuita liberamente.
    \item Una \textbf{Chiave Privata (\texttt{sk})}: Deve essere mantenuta \textbf{segreta}. È computazionalmente infattibile derivare \texttt{sk} da \texttt{pk}.
\end{itemize}

\begin{figure}[H]
    \centering
    \begin{tikzpicture}[node distance=1.5cm and 2.5cm, auto, every node/.style={text=lighttext, font=\sffamily\small}]
        % Alice
        \node[person] (alice) {Alice};
        \node[key_shape, fill=red!60!darkbg, text=white, below left=0.5cm and 0.2cm of alice, label={[xshift=-0.5cm, yshift=-0.1cm]\scriptsize Alice Privata}] (alice_priv) {Priv K};
        \node[key_shape, fill=green!60!darkbg, text=white, below right=0.5cm and 0.2cm of alice, label={[xshift=0.5cm, yshift=-0.1cm]\scriptsize Alice Pubblica}] (alice_pub) {Pub K};
        
        % Public Repository
        \node[repo, right=2cm of alice, label=above:Public Repository] (repo) {};
        \node[font=\tiny\sffamily, text=lighttext, text width=2cm, align=center, yshift=0.5cm] at (repo.center) {Bob's PUBLIC KEY\\Alice's PUBLIC KEY\\Jocelyne's PUBLIC KEY\\...};
        \node[key_shape, fill=green!60!darkbg, text=white, yshift=-0.8cm, xshift=-0.2cm, scale=0.7] at (repo.center) {Pub K}; % Simbolo chiave nel repo

        % Bob
        \node[person, right=2cm of repo] (bob) {Bob};
        \node[key_shape, fill=red!60!darkbg, text=white, below left=0.5cm and 0.2cm of bob, label={[xshift=-0.5cm, yshift=-0.1cm]\scriptsize Bob Privata}] (bob_priv) {Priv K};
        \node[key_shape, fill=green!60!darkbg, text=white, below right=0.5cm and 0.2cm of bob, label={[xshift=0.5cm, yshift=-0.1cm]\scriptsize Bob Pubblica}] (bob_pub) {Pub K};
        
        % Frecce
        \draw[dashed_arrow] (alice_pub.east) .. controls +(east:1cm) and +(west:1cm) .. (repo.west);
        \draw[dashed_arrow] (repo.east) .. controls +(east:1cm) and +(west:1cm) .. (bob_pub.west); % Bob ottiene chiave pubblica di Alice
        % Simmetricamente per Bob
        \draw[dashed_arrow, bend right=20] (bob_pub.west) to (repo.east);
    \end{tikzpicture}
    \caption{Generazione e distribuzione delle chiavi asimmetriche.}
    \label{fig:asimmetrica_keys}
\end{figure}

\subsection{Utilizzi Principali delle Chiavi Pubbliche/Private}
\begin{enumerate}[label=\arabic*.]
    \item \textbf{Confidenzialità (Cifratura):}
    \begin{itemize}
        \item Bob vuole inviare un messaggio segreto ad Alice.
        \item Bob cifra il messaggio usando la \textbf{chiave pubblica di Alice}.
        \item Alice usa la sua \textbf{chiave privata} per decifrare.
    \end{itemize}
    \textit{Esempio:} La chiave pubblica è un lucchetto aperto. Chiunque può usarlo per chiudere una scatola, ma solo chi ha la chiave (privata) può aprirla.

    \item \textbf{Autenticazione (e Firma Digitale):}
    \begin{itemize}
        \item Alice vuole inviare un messaggio a Bob, e Bob vuole essere sicuro che provenga da Alice.
        \item Alice cifra (o "firma") un messaggio (o il suo hash) usando la sua \textbf{chiave privata}.
        \item Bob usa la \textbf{chiave pubblica di Alice} per decifrare (o "verificare") la firma.
    \end{itemize}
    \textit{Esempio:} Alice "sigilla" un documento con un timbro (chiave privata) unico. Chiunque con la lente (chiave pubblica) può verificare l'autenticità del timbro.
\end{enumerate}

Relazioni tra le chiavi:
\begin{itemize}
    \item Firma: $D(\texttt{pk}_{\text{Alice}}, E(\texttt{sk}_{\text{Alice}}, \text{messaggio})) = \text{messaggio}$
    \item Confidenzialità: $D(\texttt{sk}_{\text{Alice}}, E(\texttt{pk}_{\text{Alice}}, \text{messaggio})) = \text{messaggio}$
\end{itemize}

\subsection{Differenze Chiave con la Crittografia Simmetrica}
\begin{itemize}
    \item La chiave pubblica è diversa (ma correlata) dalla chiave privata.
    \item Infattibile trovare la chiave privata dalla pubblica.
    \item Non serve distribuire una chiave segreta condivisa in anticipo.
    \item Una sola coppia di chiavi (pubblica/privata) per utente.
\end{itemize}

\subsection{Applicazioni dei Crittosistemi a Chiave Pubblica}
\begin{enumerate}
    \item \textbf{Cifratura/Decifratura:} Per confidenzialità.
    \item \textbf{Scambio di Chiavi (Key Exchange):} Per scambiare in modo sicuro una chiave di sessione simmetrica.
    \item \textbf{Firma Digitale:} Garantisce autenticazione, integrità e non ripudio.
\end{enumerate}

\subsection{Definizioni Formali}
Un sistema di crittografia a chiave pubblica è una tripla di algoritmi: $(G, E, D)$
\begin{itemize}
    \item $G()$: Algoritmo \textbf{randomizzato} che genera $(\texttt{pk}, \texttt{sk})$.
    \item $E(\texttt{pk}, m)$: Algoritmo (spesso \textbf{randomizzato}) che produce un testo cifrato $c$.
    \item $D(\texttt{sk}, c)$: Algoritmo \textbf{deterministico} che produce il messaggio $m$.
    \item \textbf{Consistenza:} $\forall (\texttt{pk}, \texttt{sk}) \leftarrow G(), \forall m \in M: D(\texttt{sk}, E(\texttt{pk}, m)) = m$.
\end{itemize}

\subsection{Funzioni One-Way a Trapdoor (TDF)}
\begin{itemize}
    \item \textbf{Definizione:} Una funzione facile da calcolare in una direzione, ma infattibile da invertire senza una "trapdoor" (chiave segreta).
    \begin{itemize}
        \item $Y = f_k(X)$: facile se $k$ (pubblico) e $X$ sono noti.
        \item $X = f_k^{-1}(Y)$: infattibile con solo $Y$ (e $k$ pubblico), facile con la trapdoor.
    \end{itemize}
    \item \textbf{Sicurezza di una TDF $(G, F, F^{-1})$:} $F(\texttt{pk}, x)$ è "one-way" senza $\texttt{sk}$.
    $\text{Pr}[\mathcal{A}(\texttt{pk}, F(\texttt{pk},x)) = x ] < \text{trascurabile}$ per un avversario $\mathcal{A}$.
\end{itemize}
\begin{figure}[H]
    \centering
    \begin{tikzpicture}[node distance=3cm and 1cm, auto]
        \node[block, text width=7em, fill=blue!50!darkbg] (chal) {Challenger (Chal)\\ $(\texttt{pk},\texttt{sk}) \leftarrow G()$\\ $x \xleftarrow{R} X$};
        \node[block, text width=7em, right=4cm of chal, fill=red!50!darkbg] (adv) {Adversary (Adv A)};
        \node[right=1cm of adv] (x_prime) {$x'$};
        
        \draw[arrow] (chal.east) -- node[midway, above, font=\scriptsize] {$\texttt{pk}, y \leftarrow F(\texttt{pk}, x)$} (adv.west);
        \draw[arrow] (adv.east) -- (x_prime);
    \end{tikzpicture}
    \caption{Modello di sicurezza per una funzione one-way (inversione).}
    \label{fig:tdf_security}
\end{figure}
\textbf{Importante:} Non si deve \textbf{mai} cifrare applicando direttamente la funzione $F$ al testo in chiaro ($c \leftarrow F(\texttt{pk}, m)$), poiché sarebbe deterministico e non semanticamente sicuro.

\section{Funzioni Hash}
\begin{itemize}
    \item \textbf{Definizione:} Prende un input di lunghezza arbitraria e produce un output di lunghezza fissa (valore hash o digest).
    \item \textbf{Algoritmo Hash One-Way:}
    \begin{itemize}
        \item Input: $m$ (stringa binaria di qualsiasi lunghezza).
        \item Output: $H(m)$ (stringa binaria di $L$ bit, es. 256 bit).
        \item Esempi: MD5 (insicuro), SHA-1 (deprecato), SHA-256, SHA-512.
    \end{itemize}
\end{itemize}
\begin{figure}[H]
    \centering
    \begin{tikzpicture}[node distance=1.5cm, auto]
        \node[document, label=above:Input] (doc_in) {Documento (lunghezza arbitraria)};
        \node[process, below=of doc_in, fill=orange!50!darkbg, text width=7em] (hash_func) {Funzione Hash};
        \node[data, below=of hash_func, label=below:Output, text width=7em, fill=green!50!darkbg] (hash_out) {Valore Hash (lunghezza fissa, es. 256 bit)};
        
        \draw[arrow] (doc_in) -- (hash_func);
        \draw[arrow] (hash_func) -- (hash_out);
    \end{tikzpicture}
    \caption{Funzionamento di una Funzione Hash.}
    \label{fig:hash_function}
\end{figure}
\textbf{Proprietà di una Buona Funzione Hash One-Way:}
\begin{enumerate}
    \item \textbf{Facile da Calcolare:} Algoritmo veloce.
    \item \textbf{Resistenza alla Preimmagine (Preimage Resistance):} Dato $h$, infattibile trovare $m$ t.c. $H(m)=h$.
    \item \textbf{Resistenza alla Seconda Preimmagine (Second Preimage Resistance):} Dato $m_1$, infattibile trovare $m_2 \neq m_1$ t.c. $H(m_1)=H(m_2)$.
    \item \textbf{Resistenza alle Collisioni (Collision Resistance):} Infattibile trovare $m_1 \neq m_2$ t.c. $H(m_1)=H(m_2)$.
    \item \textbf{Effetto Valanga (Avalanche Effect):} Piccola modifica in input $\Rightarrow$ drastico cambiamento in output.
\end{enumerate}

\section{Algoritmo RSA}
Sviluppato da \textbf{R}ivest, \textbf{S}hamir e \textbf{A}dleman (1977). Basato sulla difficoltà di fattorizzare grandi numeri.

\subsection{Permutazioni a Trapdoor}
RSA è un esempio. Tre algoritmi: $(G, F, F^{-1})$.
\begin{itemize}
    \item $G$: genera $(\texttt{pk}, \texttt{sk})$.
    \item $\texttt{pk}$ definisce $F(\texttt{pk}, \cdot): X \to X$.
    \item $F(\texttt{pk}, x)$: valuta la funzione.
    \item $F^{-1}(\texttt{sk}, y)$: inverte la funzione usando $\texttt{sk}$.
\end{itemize}
\textbf{Permutazione a Trapdoor Sicura:} $F(\texttt{pk}, \cdot)$ è one-way senza la trapdoor $\texttt{sk}$.

\subsection{Basi Matematiche di RSA}
\begin{itemize}
    \item $N = p \cdot q$, con $p, q$ primi grandi.
    \item $Z_N = \{0, 1, \dots, N-1\}$.
    \item $(Z_N)^* = \{x \in Z_N \mid \text{mcd}(x, N) = 1\}$.
    \item Funzione Totiente di Eulero: $\phi(N) = (p-1)(q-1) = N - p - q + 1$.
    \item Teorema di Eulero: $\forall x \in (Z_N)^*, x^{\phi(N)} \equiv 1 \pmod N$.
\end{itemize}

\subsection{Generazione delle Chiavi RSA (Algoritmo $G$)}
\begin{enumerate}
    \item Scegliere $p, q$ primi grandi e distinti.
    \item Calcolare $N = p \cdot q$.
    \item Calcolare $\phi(N) = (p-1)(q-1)$.
    \item Scegliere $e$ (esponente pubblico) t.c. $1 < e < \phi(N)$ e $\text{mcd}(e, \phi(N)) = 1$. (Es. $e=65537$).
    \item Calcolare $d$ (esponente privato) t.c. $e \cdot d \equiv 1 \pmod{\phi(N)}$. ($d$ è l'inverso moltiplicativo di $e \pmod{\phi(N)}$).
\end{enumerate}
\begin{itemize}
    \item \textbf{Chiave Pubblica:} $\texttt{pk} = (N, e)$.
    \item \textbf{Chiave Privata:} $\texttt{sk} = (N, d)$ (o $(p, q, d)$). $p, q, \phi(N)$ segreti.
\end{itemize}

\subsection{Cifratura e Decifratura RSA ("Textbook RSA")}
Sia $m$ il messaggio ($0 \le m < N$).
\begin{itemize}
    \item \textbf{Cifratura:} $c = m^e \pmod N$.
    \item \textbf{Decifratura:} $m = c^d \pmod N$.
\end{itemize}
\textit{Perché funziona:} $c^d \equiv (m^e)^d \equiv m^{ed} \pmod N$. Poiché $ed \equiv 1 \pmod{\phi(N)}$, $ed = k \cdot \phi(N) + 1$.
Quindi $m^{ed} \equiv m^{k\phi(N)+1} \equiv (m^{\phi(N)})^k \cdot m^1 \equiv 1^k \cdot m \equiv m \pmod N$ (per il Teorema di Eulero).

\subsection{Esempio RSA (Piccolo)}
\begin{enumerate}
    \item \textbf{Generazione Chiavi (Bob):}
    \begin{itemize}
        \item $p=5, q=11 \Rightarrow N = 55, \phi(N) = (5-1)(11-1) = 40$.
        \item Sceglie $e=3$ (mcd(3, 40)=1).
        \item Calcola $d$ t.c. $3d \equiv 1 \pmod{40} \Rightarrow d=27$ (poiché $3 \cdot 27 = 81 \equiv 1 \pmod{40}$).
        \item $\texttt{pk}_{\text{Bob}} = (55, 3)$, $\texttt{sk}_{\text{Bob}} = (55, 27)$.
    \end{itemize}
    \item \textbf{Cifratura (Alice invia $m=13$ a Bob):}
    \begin{itemize}
        \item $c = 13^3 \pmod{55} = 2197 \pmod{55}$.
        \item $2197 = 39 \cdot 55 + 52 \Rightarrow c=52$. Alice invia $c=52$.
    \end{itemize}
    \item \textbf{Decifratura (Bob riceve $c=52$):}
    \begin{itemize}
        \item $m = 52^{27} \pmod{55}$.
        \item $52 \equiv -3 \pmod{55}$.
        \item $m \equiv (-3)^{27} \pmod{55}$.
        \item $m \equiv (-3^{3})^9 \equiv (-27)^9 \equiv (28)^9 \pmod{55}$.
        \item $28^2 = 784 \equiv 14 \pmod{55}$.
        \item $28^3 \equiv 14 \cdot 28 = 392 \equiv 7 \pmod{55}$.
        \item $28^9 = (28^3)^3 \equiv 7^3 = 343 \equiv 13 \pmod{55}$.
        \item Bob ottiene $m=13$.
    \end{itemize}
\end{enumerate}

\subsection{Sicurezza di RSA}
\begin{itemize}
    \item \textbf{Assunzione RSA:} Difficile calcolare $m$ da $c=m^e \pmod N$ senza conoscere $d$ o la fattorizzazione di $N$.
    \item Rompere RSA è comunemente legato alla \textbf{fattorizzazione di $N$}.
    \item \textbf{"Textbook RSA" è insicuro per la cifratura diretta:}
    \begin{itemize}
        \item \textbf{Deterministico:} Non semanticamente sicuro.
        \item Vulnerabile a vari attacchi (es. se $m$ è piccolo).
        \item \textbf{Soluzione:} Usare schemi di padding (es. OAEP) o schemi ibridi.
        La permutazione a trapdoor RSA \textbf{non è} uno schema di cifratura di per sé.
    \end{itemize}
\end{itemize}
Schema di cifratura RSA (ISO standard) usando una funzione hash $H$ e uno schema di cifratura simmetrica $(E_s, D_s)$:
\begin{itemize}
    \item $G()$: genera parametri RSA $\texttt{pk}=(N,e), \texttt{sk}=(N,d)$.
    \item $E(\texttt{pk}, m)$:
        \begin{enumerate}
            \item Scegli $x \in Z_N$ casuale.
            \item $y \leftarrow \text{RSA}(x) = x^e \pmod N$.
            \item $k \leftarrow H(x)$.
            \item Output $(y, E_s(k,m))$.
        \end{enumerate}
    \item $D(\texttt{sk}, (y,c))$:
        \begin{enumerate}
            \item $x \leftarrow \text{RSA}^{-1}(y) = y^d \pmod N$.
            \item $k \leftarrow H(x)$.
            \item Output $D_s(k,c)$.
        \end{enumerate}
\end{itemize}


\subsection{Prestazioni e Pratica di RSA}
\begin{itemize}
    \item \textbf{Scelta di $e$ piccolo} (es. $e=65537 = 2^{16}+1$) per cifratura veloce.
    \item \textbf{Non scegliere $d$ troppo piccolo:}
    \begin{itemize}
        \item Wiener'87: Se $d < N^{0.25}$, RSA insicuro.
        \item BD'98: Se $d < N^{0.292}$, RSA insicuro.
    \end{itemize}
    \item \textbf{Lunghezze delle Chiavi RSA vs. Simmetriche (Sicurezza Equivalente):}
    \begin{table}[H]
        \centering
        \begin{tabular}{|l|l|}
            \hline
            \rowcolor{bg_custom}
            \textbf{\sffamily Forza Simmetrica (bit)} & \textbf{\sffamily Dimensione Modulo RSA (bit)} \\ \hline \hline
            80                               & 1024                              \\ \hline
            128                              & 3072                              \\ \hline
            256 (AES)                        & 15360                             \\ \hline
        \end{tabular}
        \caption{Lunghezze di chiave RSA raccomandate.}
        \label{tab:rsa_keys}
    \end{table}
    \item \textbf{Attacchi di Implementazione (Side-channel):}
    \begin{itemize}
        \item \textit{Timing attack}: Il tempo di decifratura può rivelare $d$.
        \item \textit{Power attack}: Il consumo energetico può rivelare $d$.
        \item \textit{Fault attack}: Errori indotti nel calcolo possono rivelare $d$.
        \item \textit{Difese:} Blinding, verifica output.
    \end{itemize}
    \item \textbf{Problemi nella Generazione delle Chiavi (es. OpenSSL):}
    Se il PRNG ha poca entropia, diversi dispositivi possono generare lo stesso primo $p$.
    Se $N_1 = p \cdot q_1$ e $N_2 = p \cdot q_2$, allora $\text{mcd}(N_1, N_2) = p$, rompendo entrambe le chiavi.
    \textbf{Lezione:} Assicurare buona entropia per il PRNG.
\end{itemize}

\section{Firma Digitale}
Fornisce autenticazione, integrità e non ripudio.

\subsection{Processo Generale di Firma Digitale}
\begin{figure}[H]
    \centering
    \begin{tikzpicture}[
        node distance=1.5cm,
        box/.style={draw, rectangle, minimum width=2cm, minimum height=1cm},
        >=stealth
    ]
        % Left side (Bob - Signer)
        \node[draw, circle] (bob) at (0,0) {Bob};
        \node[box] (msg) at (0,-2) {M};
        \node[box] (hash) at (2,-2) {Hash};
        \node[box] (sign) at (4,-2) {Sign};
        \node[box] (sk) at (4,-4) {sk$_{\text{Bob}}$};
        
        % Arrows for Bob's side
        \draw[->] (msg) -- (hash);
        \draw[->] (hash) -- (sign);
        \draw[->] (sk) -- (sign);
        
        % Network transmission
        \node[draw, cloud] (network) at (6,-2) {(M, S)};
        \draw[->] (sign) -- (network);
        
        % Right side (Alice - Verifier)
        \node[draw, circle] (alice) at (12,0) {Alice};
        \node[box] (msg2) at (8,-2) {M};
        \node[box] (hash2) at (10,-2) {Hash};
        \node[box] (verify) at (12,-2) {Verify};
        \node[box] (pk) at (12,-4) {pk$_{\text{Bob}}$};
        \node[box] (result) at (14,-2) {Valid/Invalid};
        
        % Arrows for Alice's side
        \draw[->] (network) -- (msg2);
        \draw[->] (msg2) -- (hash2);
        \draw[->] (hash2) -- (verify);
        \draw[->] (pk) -- (verify);
        \draw[->] (verify) -- (result);
        
        % Labels
        \node at (0,1) {Firmatario};
        \node at (12,1) {Verificatore};
    \end{tikzpicture}
    \caption{Processo di Firma e Verifica Digitale (Versione Semplificata).}
    \label{fig:digital_signature}
\end{figure}
\begin{enumerate}
    \item \textbf{Mittente (Bob):}
    \begin{itemize}
        \item Calcola $h = H(M)$ (hash del messaggio $M$).
        \item "Cifra" $h$ con la sua chiave privata $\texttt{sk}_{\text{Bob}}$: $S = \text{Sign}(\texttt{sk}_{\text{Bob}}, h)$.
        \item Invia $(M, S)$.
    \end{itemize}
    \item \textbf{Destinatario (Alice):}
    \begin{itemize}
        \item Calcola $h' = H(M_{\text{ricevuto}})$.
        \item "Decifra" $S$ con la chiave pubblica $\texttt{pk}_{\text{Bob}}$: $h_{\text{verificato}} = \text{Verify}(\texttt{pk}_{\text{Bob}}, S)$.
        \item Se $h' == h_{\text{verificato}}$, la firma è valida.
    \end{itemize}
\end{enumerate}

\subsection{Firma Digitale basata su RSA}
\begin{itemize}
    \item \textbf{Firma (Bob):}
    \begin{enumerate}
        \item Calcola $h = H(M)$.
        \item Calcola $S = h^d \pmod N$ (dove $(N,d)$ è $\texttt{sk}_{\text{Bob}}$).
        (Nota: in pratica si usa padding su $h$, es. PSS).
    \end{enumerate}
    \item \textbf{Verifica (Alice):}
    \begin{enumerate}
        \item Ottiene $\texttt{pk}_{\text{Bob}} = (N,e)$.
        \item Calcola $h_{\text{recuperato}} = S^e \pmod N$.
        \item Calcola $h_{\text{locale}} = H(M_{\text{ricevuto}})$.
        \item Se $h_{\text{recuperato}} == h_{\text{locale}}$, firma valida.
    \end{enumerate}
\end{itemize}

\subsection{Esempio di Firma RSA (Piccolo)}
$\texttt{pk}_{\text{Bob}} = (55, 3)$, $\texttt{sk}_{\text{Bob}} = (55, 27)$.
\begin{enumerate}
    \item \textbf{Bob firma $m_{\text{doc}}=19$ (direttamente, per semplicità):}
    \begin{itemize}
        \item $S = 19^{27} \pmod{55} = 24$.
        \item Bob invia $(m_{\text{doc}}=19, S=24)$.
    \end{itemize}
    \item \textbf{Cathy verifica $(m_{\text{doc}}=19, S=24)$:}
    \begin{itemize}
        \item $t = S^e \pmod N = 24^3 \pmod{55} = 13824 \pmod{55} = 19$.
        \item Poiché $t=19 == m_{\text{doc}}=19$, la firma è valida.
    \end{itemize}
\end{enumerate}

\subsection{Perché Firmare l'Hash e non l'Intero Documento?}
\begin{enumerate}
    \item \textbf{Efficienza:} Firmare $H(M)$ (piccolo) è più veloce che firmare $M$ (grande).
    \item \textbf{Compatibilità:} $M$ potrebbe essere $>N$. $H(M)$ è più gestibile.
    \item \textbf{Sicurezza:} Firmare l'hash (con padding) è più robusto.
\end{enumerate}
\begin{figure}[H]
    \centering
    \begin{tikzpicture}[
        node distance=2cm,
        box/.style={draw, rectangle, minimum width=2cm, minimum height=1cm},
        >=stealth
    ]
        % Left side (Bob)
        \node[draw, circle] (bob) at (0,0) {Bob};
        \node[box] (doc) at (0,-2) {Documento};
        \node[box] (hash) at (3,-2) {Hash};
        \node[box] (sign) at (6,-2) {Firma};
        \node[box] (sk) at (6,-4) {Chiave Privata};
        
        % Network
        \node[draw, cloud] (network) at (8,-2) {Rete};
        
        % Right side (Cathy)
        \node[draw, circle] (cathy) at (14,0) {Cathy};
        \node[box] (doc2) at (10,-2) {Documento};
        \node[box] (hash2) at (12,-2) {Hash};
        \node[box] (verify) at (14,-2) {Verifica};
        \node[box] (pk) at (14,-4) {Chiave Pubblica};
        \node[box] (result) at (16,-2) {Valido/Non Valido};
        
        % Arrows
        \draw[->] (doc) -- (hash);
        \draw[->] (hash) -- (sign);
        \draw[->] (sk) -- (sign);
        \draw[->] (sign) -- (network);
        
        \draw[->] (network) -- (doc2);
        \draw[->] (doc2) -- (hash2);
        \draw[->] (hash2) -- (verify);
        \draw[->] (pk) -- (verify);
        \draw[->] (verify) -- (result);
        
        % Labels
        \node at (0,1) {Firmatario};
        \node at (14,1) {Verificatore};
    \end{tikzpicture}
    \caption{Firma Digitale per Documenti Lunghi (Versione Semplificata).}
    \label{fig:long_doc_signature}
\end{figure}

\subsection{Vantaggi della Firma Digitale}
\begin{itemize}
    \item \textbf{Non Falsificabile (Unforgeable).}
    \item \textbf{Non Ripudiabile (Undeniable).}
    \item \textbf{Verificabile Universalmente (Universally Verifiable).}
    \item \textbf{Specifica per il Documento.}
\end{itemize}

\subsection{Riassunto Passaggi Firma Digitale}
\begin{enumerate}
    \item \textbf{Setup delle Chiavi Pubbliche e Private} (una tantum).
    \item \textbf{Firma di un Documento} (con chiave privata).
    \item \textbf{Verifica di una Firma} (con chiave pubblica del firmatario; il verificatore non necessita di proprie chiavi).
\end{enumerate}

\section{Certificati Digitali}
(Brevemente accennato nelle slide)
\begin{itemize}
    \item \textbf{Problema:} Come sapere se una $\texttt{pk}$ appartiene veramente ad Alice?
    \item \textbf{Soluzione: Certificati Digitali (es. X.509).}
    Un certificato lega un'identità a una $\texttt{pk}$, ed è firmato da una \textbf{Certificate Authority (CA)} fidata.
    Questo crea una catena di fiducia (Public Key Infrastructure - PKI).
\end{itemize}

\section{Altri Crittosistemi Asimmetrici}

\subsection{ElGamal Cryptosystem}
Basato sul protocollo Diffie-Hellman e sulla difficoltà del logaritmo discreto.
\begin{itemize}
    \item \textbf{Parametri Pubblici:} Primo $p$, generatore $g \pmod p$.
    \item \textbf{Generazione Chiavi (Bob):}
    \begin{itemize}
        \item Sceglie privato $a \in [0, p-2]$. Calcola $A = g^a \pmod p$.
        \item $\texttt{pk}_{\text{Bob}} = (p, g, A)$, $\texttt{sk}_{\text{Bob}} = a$.
    \end{itemize}
    \item \textbf{Cifratura (Alice invia $m$ a Bob):}
    \begin{itemize}
        \item Sceglie $b \in [0, p-2]$ casuale (effimero).
        \item Calcola $B = g^b \pmod p$.
        \item Calcola $c = A^b \cdot m \pmod p$.
        \item Testo cifrato: $(B, c)$.
    \end{itemize}
    \item \textbf{Decifratura (Bob riceve $(B,c)$):}
    \begin{itemize}
        \item Calcola $x = p-1-a$.
        \item $m = B^x \cdot c \pmod p$.
    \end{itemize}
    \textit{Perché funziona:} $B^x c \equiv (g^b)^{p-1-a} (g^a)^b m \equiv g^{b(p-1)-ab+ab} m \equiv (g^{p-1})^b m \equiv 1^b m \equiv m \pmod p$.
    \item \textbf{Caratteristiche:} Randomizzato, espansione del messaggio (testo cifrato doppio del chiaro). Sicurezza basata sul logaritmo discreto.
\end{itemize}

\subsection{Rabin Cryptosystem}
Basato sulla difficoltà di calcolare radici quadrate modulo $N=pq$.
\begin{itemize}
    \item \textbf{Generazione Chiavi (Bob):}
    \begin{itemize}
        \item $p, q$ primi (spesso $p, q \equiv 3 \pmod 4$). $N = pq$.
        \item $\texttt{pk}_{\text{Bob}} = N$, $\texttt{sk}_{\text{Bob}} = (p, q)$.
    \end{itemize}
    \item \textbf{Cifratura (Alice invia $m$):}
    \begin{itemize}
        \item $c = m^2 \pmod N$.
    \end{itemize}
    \item \textbf{Decifratura (Bob riceve $c$):}
    \begin{itemize}
        \item Calcola le 4 radici quadrate di $c \pmod N$ usando $p,q$ e il Teorema Cinese del Resto. Una di queste è $m$.
        Se $p \equiv 3 \pmod 4$, radici $\pmod p$ sono $\pm c^{(p+1)/4} \pmod p$. Simile per $q$.
    \end{itemize}
    \item \textbf{Caratteristiche:} Cifratura efficiente. Rompere Rabin è provabilmente difficile quanto fattorizzare $N$. Ambiguità nella decifratura.
\end{itemize}

\subsection{Elliptic Curve Cryptography (ECC)}
Utilizza curve ellittiche su campi finiti.
\begin{itemize}
    \item Sicurezza comparabile a RSA con chiavi \textbf{molto più piccole}.
    \item Ideale per dispositivi con risorse limitate.
\end{itemize}

\subsection{Tabella Comparativa delle Dimensioni delle Chiavi (NIST SP-800-57)}
Per un livello di sicurezza computazionale equivalente:
\begin{table}[H]
    \centering
    \renewcommand{\arraystretch}{1.2} % Aumenta lo spazio tra le righe
    \begin{tabular}{|p{2.5cm}|p{3.5cm}|p{2.5cm}|p{2.5cm}|}
        \hline
        \rowcolor{bg_custom}
        \textbf{\sffamily Symmetric Key Algorithms (bits)} & \textbf{\sffamily Diffie-Hellman, Digital Signature Algorithm (L=pk, N=sk bits)} & \textbf{\sffamily RSA (size of $n$ in bits)} & \textbf{\sffamily ECC (modulus size in bits)} \\ \hline \hline
        80  & L = 1024 \newline N = 160 & 1024  & 160-223 \\ \hline
        112 & L = 2048 \newline N = 224 & 2048  & 224-255 \\ \hline
        128 & L = 3072 \newline N = 256 & 3072  & 256-383 \\ \hline
        192 & L = 7680 \newline N = 384 & 7680  & 384-511 \\ \hline
        256 & L = 15,360 \newline N = 512 & 15,360 & 512+    \\ \hline
    \end{tabular}
    \caption{Dimensioni comparabili delle chiavi per diversi algoritmi.}
    \label{tab:key_sizes_comparison}
    \small\textit{Note: L = size of public key, N = size of private key (per DH/DSA).}
\end{table}

\end{document}