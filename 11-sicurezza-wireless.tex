\documentclass{article}

% --- Encoding e lingua ---
\usepackage[utf8]{inputenc}
\usepackage[italian]{babel}

% --- Margini e layout ---
\usepackage{geometry}
\geometry{a4paper, margin=1in}

% --- Font sans-serif ---
\usepackage[scaled]{helvet}
\renewcommand{\familydefault}{\sfdefault}
\usepackage[T1]{fontenc}

% --- Matematica ---
\usepackage{amsmath}
\usepackage{amssymb}

% --- Liste personalizzate ---
\usepackage{enumitem}

% --- Immagini ---
\usepackage{float}
\usepackage{tikz} % per disegni
\usepackage{graphicx} % per immagini
\usetikzlibrary{shapes.geometric, positioning, arrows.meta, calc, backgrounds}

% --- Hyperlink ---
\usepackage{hyperref}
\hypersetup{
    pdftitle={Appunti sulla Sicurezza delle Reti Wireless},
    colorlinks=true,
    linkcolor=cyan,
    filecolor=magenta,      
    urlcolor=green,
    citecolor=red,
    bookmarksopen=true,
    bookmarksnumbered=true
}

% --- Colori e sfondo nero ---
\usepackage{xcolor}
\definecolor{darkbg}{RGB}{30,30,30} % Un nero non assoluto
\definecolor{lighttext}{RGB}{230,230,230} % Un bianco non accecante
\definecolor{highlight}{RGB}{100,150,255} % Un blu per evidenziare

\pagecolor{darkbg}
\color{lighttext}

% Stili per TikZ (Dark Theme)
\tikzstyle{block_dark} = [rectangle, draw=lighttext, fill=gray!60!darkbg, text=lighttext, 
    minimum height=2em, minimum width=6em, text centered, rounded corners]
\tikzstyle{arrow_dark} = [-{Stealth[length=3mm, width=2mm]}, draw=lighttext, thick]
\tikzstyle{line_dark} = [draw=lighttext, thick]
\tikzstyle{node_dark} = [circle, draw=lighttext, fill=gray!40!darkbg, text=lighttext, minimum size=7mm]
\tikzstyle{label_dark} = [text=lighttext, font=\footnotesize]

% --- Evidenziazione del codice (richiede -shell-escape) ---
% Compilare con: pdflatex -shell-escape nomefile.tex
\usepackage{minted}
\setminted{
    frame=lines,
    framesep=2mm,
    fontsize=\small,
    breaklines=true,
    style=monokai, % Stile di highlighting dark-friendly
    bgcolor=black!80 % Sfondo leggermente diverso per i blocchi minted
}
\usemintedstyle{monokai}


% --- Titolo ---
\title{Appunti sulla Sicurezza delle Reti Wireless}
\author{Basato sulle slide del Prof. Jocelyne Elias}
\date{\today}

\begin{document}

\maketitle
\tableofcontents
\newpage

% Contenuto qui

\section{L'Ambiente Wireless/Radio}
L'ambiente wireless è intrinsecamente più vulnerabile di quello cablato. I punti di attacco principali sono tre:

\begin{itemize}
    \item \textbf{Client Wireless:} Qualsiasi dispositivo che si connette in modalità wireless (es. smartphone, laptop, tablet con Wi-Fi, sensori wireless, dispositivi Bluetooth).
    \begin{itemize}
        \item \textit{Esempio pratico:} Il tuo smartphone che si connette alla rete Wi-Fi di casa.
    \end{itemize}
    \item \textbf{Punto di Accesso Wireless (AP):} Fornisce la connessione alla rete o a un servizio (es. ripetitori cellulari (Base Station), hotspot Wi-Fi, AP per reti LAN o WAN).
    \begin{itemize}
        \item \textit{Esempio pratico:} Il router Wi-Fi a casa tua.
    \end{itemize}
    \item \textbf{Mezzo di Trasmissione:} Le onde radio usate per trasmettere i dati. Essendo un mezzo condiviso e aperto, è una fonte significativa di vulnerabilità.
    \begin{itemize}
        \item \textit{Esempio pratico:} L'aria attraverso cui viaggiano i segnali Wi-Fi, Bluetooth, ecc.
    \end{itemize}
\end{itemize}

\begin{figure}[H]
    \centering
    \begin{tikzpicture}[node distance=3cm, auto, every node/.style={text=lighttext}]
        \node[block_dark] (endpoint) {Endpoint (Client)};
        \node[block_dark, right=of endpoint] (medium) {Wireless Medium};
        \node[block_dark, right=of medium] (ap) {Access Point};

        \draw[arrow_dark, <->] (endpoint) -- node[above, midway] {Segnali Radio} (medium);
        \draw[arrow_dark, <->] (medium) -- node[above, midway] {Segnali Radio} (ap);
        
        \begin{pgfonlayer}{background}
            \node[rectangle, fill=darkbg!80, inner sep=1cm, fit=(endpoint)(medium)(ap), rounded corners] {};
        \end{pgfonlayer}
    \end{tikzpicture}
    \caption{Componenti di una Rete Wireless.}
    \label{fig:wireless_components}
\end{figure}


\subsection{Fattori Chiave di Rischio nelle Reti Wireless}
(rispetto alle Reti Cablate):
\begin{itemize}
    \item \textbf{Canale Wireless:}
    \begin{itemize}
        \item Comunicazioni \textbf{broadcast}: Chiunque nel raggio d'azione può "ascoltare".
        \item Più soggetto a \textbf{intercettazioni} e \textbf{interferenze}.
        \item Vulnerabile ad \textbf{attacchi attivi} che sfruttano le debolezze dei protocolli di comunicazione.
    \end{itemize}
    \item \textbf{Mobilità:}
    \begin{itemize}
        \item I dispositivi wireless sono portatili e mobili.
        \item Comporta rischi come smarrimento, furto, connessione a reti insicure.
        \item \textit{Esempio pratico:} Perdere lo smartphone aziendale che contiene dati sensibili.
    \end{itemize}
    \item \textbf{Risorse Limitate:}
    \begin{itemize}
        \item Dispositivi come smartphone e tablet hanno risorse di memoria ed elaborazione limitate.
        \item Questo li rende meno capaci di contrastare minacce complesse (es. DoS, malware).
    \end{itemize}
    \item \textbf{Accessibilità Fisica:}
    \begin{itemize}
        \item Dispositivi come sensori IoT o robot possono essere lasciati incustoditi in luoghi remoti o ostili.
        \item Aumenta la vulnerabilità ad \textbf{attacchi fisici} (manomissione, estrazione di dati).
        \item \textit{Esempio pratico:} Un sensore ambientale in un campo agricolo potrebbe essere facilmente manomesso da un malintenzionato.
    \end{itemize}
\end{itemize}

\section{Possibili Minacce alla Rete Wireless}
\begin{itemize}
    \item \textbf{Associazione Dannosa (Malicious Association / Rogue AP):}
    \begin{itemize}
        \item Un dispositivo wireless malevolo si configura per apparire come un AP legittimo (es. \texttt{"Free\_Airport\_WiFi"}).
        \item Induce gli utenti a connettersi, permettendo all'attaccante di rubare password o intercettare dati.
        \item Può essere usato per penetrare una rete cablata attraverso l'AP wireless legittimo compromesso.
    \end{itemize}
    \item \textbf{Reti Ad Hoc:}
    \begin{itemize}
        \item Reti peer-to-peer tra dispositivi wireless senza un AP centrale.
        \item La mancanza di un punto di controllo centrale le rende una minaccia alla sicurezza.
    \end{itemize}
    \item \textbf{Reti Non Tradizionali:}
    \begin{itemize}
        \item Dispositivi come Bluetooth, scanner di codici a barre, PDA.
        \item Presentano rischi di intercettazione e spoofing.
    \end{itemize}
    \item \textbf{Furto di Identità (MAC Spoofing):}
    \begin{itemize}
        \item Un attaccante intercetta il traffico di rete, identifica l'indirizzo MAC di un computer con privilegi.
        \item L'attaccante poi configura il proprio dispositivo con quel MAC address per impersonare l'utente legittimo.
    \end{itemize}
    \item \textbf{Attacchi Man-in-the-Middle (MitM):}
    \begin{itemize}
        \item L'attaccante si interpone tra un utente e un AP (o tra due utenti).
        \item Entrambe le parti credono di comunicare direttamente, ma in realtà la comunicazione passa attraverso l'attaccante.
        \item Le reti wireless sono particolarmente vulnerabili.
    \end{itemize}
    \item \textbf{Denial of Service (DoS):}
    \begin{itemize}
        \item L'attaccante bombarda un AP o un client con traffico eccessivo o malformato.
        \item L'obiettivo è consumare le risorse del target, rendendolo inutilizzabile.
        \item Facile da realizzare in ambiente wireless.
    \end{itemize}
    \item \textbf{Iniezione di Rete (Network Injection):}
    \begin{itemize}
        \item Mira a AP esposti a traffico di rete non filtrato.
        \item L'attaccante invia comandi di riconfigurazione fasulli per degradare le prestazioni della rete.
    \end{itemize}
\end{itemize}

\subsection{Tipi di Attacchi (Generale)}
\begin{itemize}
    \item \textbf{Passivo:} L'attaccante osserva/registra il traffico senza modificarlo (es. eavesdropping, analisi del traffico).
    \item \textbf{Attivo:} L'attaccante modifica il traffico, si impersona o crea nuovo traffico (es. masquerade, modifica dati, replay, DoS).
\end{itemize}

\begin{figure}[H]
    \centering
    \begin{tikzpicture}[node distance=1.5cm and 2.5cm, every node/.style={text=lighttext}]
        % Normal Flow
        \node[block_dark] (c1) {Client};
        \node[block_dark, right=of c1] (s1) {Server};
        \draw[arrow_dark] (c1) -- (s1);
        \node[label_dark, below=0.1cm of c1, xshift=1.25cm] {(a) Normal flow};

        % Passive Attack
        \node[block_dark, below=of c1, yshift=-0.5cm] (c2) {Client};
        \node[block_dark, right=of c2] (s2) {Server};
        \node[block_dark, below=of c2, xshift=1.25cm, yshift=0.3cm] (atk2) {Attacker};
        \draw[arrow_dark] (c2) -- (s2);
        \draw[arrow_dark, dashed] (c2) -- (atk2);
        \draw[arrow_dark, dashed] (s2) -- (atk2);
        \node[label_dark, below=0.1cm of atk2] {(b) Passive attack};
        
        % Active Attack (Masquerade, Data Mod)
        \node[block_dark, right=of s1, xshift=1.5cm] (c3) {Client};
        \node[block_dark, right=of c3] (s3) {Server};
        \node[block_dark, below=of c3, xshift=1.25cm, yshift=0.3cm] (atk3) {Attacker};
        \draw[arrow_dark] (c3) -- (atk3);
        \draw[arrow_dark] (atk3) -- (s3);
        \node[label_dark, below=0.1cm of atk3] {(c) Active (masquerade, mod)};

        % Active Attack (Masquerade, Data Mod, DoS)
        \node[block_dark, below=of c2, yshift=-2cm] (c4) {Client};
        \node[block_dark, right=of c4] (s4) {Server};
        \node[block_dark, below=of c4, xshift=1.25cm, yshift=0.3cm] (atk4) {Attacker};
        \draw[arrow_dark] (atk4) -- (s4); % Attacker sends to server
        % No direct line from client in this common representation
        \node[label_dark, below=0.1cm of atk4] {(d) Active (masquerade, DoS)};

        % Active Attack (Replay)
        \node[block_dark, right=of s3, xshift=1.5cm] (c5) {Client};
        \node[block_dark, right=of c5] (s5) {Server};
        \node[block_dark, below=of c5, xshift=1.25cm, yshift=0.3cm] (atk5) {Attacker};
        \draw[arrow_dark, dashed] (c5) -- (atk5); % Capture
        \draw[arrow_dark] (atk5) -- (s5);    % Replay
        \node[label_dark, below=0.1cm of atk5] {(e) Active (replay)};
        
        \begin{pgfonlayer}{background}
            \node[rectangle, fill=darkbg!80, inner sep=0.5cm, fit=(c1)(s1), rounded corners] {};
            \node[rectangle, fill=darkbg!80, inner sep=0.5cm, fit=(c2)(s2)(atk2), rounded corners] {};
            \node[rectangle, fill=darkbg!80, inner sep=0.5cm, fit=(c3)(s3)(atk3), rounded corners] {};
            \node[rectangle, fill=darkbg!80, inner sep=0.5cm, fit=(c4)(s4)(atk4), rounded corners] {};
            \node[rectangle, fill=darkbg!80, inner sep=0.5cm, fit=(c5)(s5)(atk5), rounded corners] {};
        \end{pgfonlayer}
    \end{tikzpicture}
    \caption{Tipi di Attacchi alla Sicurezza.}
    \label{fig:security_attacks}
\end{figure}


\section{Sicurezza a Livello Radio}
\subsection{Denial of Service (Jamming)}
\begin{itemize}
    \item Consiste nell'\textbf{emettere un forte segnale radio} (rumore) sulla stessa frequenza utilizzata dalla rete target.
    \item Questo "disturba" o "copre" i segnali legittimi, impedendo la comunicazione.
    \item \textbf{Contromisura:}
    \begin{itemize}
        \item Usare \textbf{più frequenze portanti (banda larga)}.
        \item \textbf{Frequency Hopping Spread Spectrum (FHSS):}
        \begin{itemize}
            \item Si selezionano più portanti e si "salta" da una all'altra secondo una sequenza pseudocasuale.
            \item Se una singola portante è disturbata, si perde solo una piccola parte dell'informazione.
            \item \textit{Esempio pratico:} Il Bluetooth utilizza FHSS.
        \end{itemize}
    \end{itemize}
\end{itemize}
(La slide 9 mostra un diagramma FHSS: (a) Channel assignment con barre di energia per diverse frequenze $f_1..f_8$ e una sequenza di uso, (b) Channel use che mostra i salti di frequenza nel tempo. Questo illustra come l'energia sia distribuita su più canali e come il sistema cambi canale rapidamente).

\subsection{Eavesdropping (Intercettazione) e Sniffing}
\begin{itemize}
    \item Relativamente semplice intercettare le trasmissioni radio.
    \item Se si usa FHSS, conoscere il "seed" del PRNG permette di seguire le trasmissioni.
    \item \textbf{Cosa serve all'attaccante:}
    \begin{enumerate}
        \item Un \textbf{ricevitore} (con antenna adeguata) alla corretta distanza.
        \item Conoscenza della \textbf{modulazione} utilizzata.
        \item Conoscenza del \textbf{protocollo} utilizzato.
    \end{enumerate}
\end{itemize}

\subsection{Replay Attack (Attacco di Ripetizione)}
\begin{itemize}
    \item Un attaccante cattura una trasmissione legittima e la \textbf{ritrasmette successivamente}.
    \item Può essere effettuato anche \textbf{senza comprendere il contenuto} del messaggio.
    \item \textit{Esempio pratico:} Catturare il segnale del telecomando di un cancello (OOK su 433MHz) e ritrasmetterlo.
    \item \textbf{Importante:} Meccanismi di sola confidenzialità e integrità NON proteggono da replay.
\end{itemize}
\textbf{Contromisure al Replay Attack:}
\begin{enumerate}
    \item \textbf{Rolling Code (Codice Progressivo):}
    \begin{itemize}
        \item Il trasmettitore e il ricevitore condividono un PRNG sincronizzato.
        \item Ad ogni trasmissione, viene inviato un codice diverso.
        \item Il ricevitore accetta un codice solo se "atteso" o entro una "finestra".
        \item \textit{Problema:} Disallineamento se trasmissioni perse; mitigato dalla finestra.
        \item \textit{Efficacia:} Rende il replay meno efficiente.
    \end{itemize}
    \item \textbf{Challenge and Response (Sfida-Risposta):}
    \begin{itemize}
        \item Richiede un canale di ritorno (transceiver).
        \item \textbf{Funzionamento:}
        \begin{enumerate}
            \item L'\textbf{autenticando} richiede accesso.
            \item L'\textbf{autenticatore} genera un numero casuale (nonce) e lo invia.
            \item L'autenticando cifra il nonce e lo reinvia.
            \item L'autenticatore decifra e valida.
        \end{enumerate}
        \item Previene il replay perché il nonce cambia.
    \end{itemize}
\end{enumerate}

\subsection{Cifratura}
\begin{itemize}
    \item Usata per \textbf{confidenzialità} e \textbf{integrità} dei dati.
    \item Nel mondo radio:
    \begin{itemize}
        \item \textbf{Cifrari a blocco:} Es. AES.
        \item \textbf{Cifrari a flusso:} Es. Kasumi (A5/3 "rotto" nel 2010).
    \end{itemize}
\end{itemize}

\section{Radio Frequency Identification (RFID)}
\begin{itemize}
    \item Sistema di identificazione automatica tramite campi elettromagnetici.
    \item \textbf{Componenti:} Tag RFID, Lettore RFID.
    \item \textbf{Funzionamento:} Il lettore interroga, il tag risponde con i dati.
    \item \textit{Esempio pratico:} Badge Unibo (EM410X con ID univoco).
\end{itemize}

\subsection{Vulnerabilità RFID}
\begin{enumerate}
    \item \textbf{Cloning:}
    \begin{itemize}
        \item \textbf{Come aggirare:} Acquistare tag riscrivibili e copiare l'ID.
    \end{itemize}
    \item \textbf{Spoofing:}
    \begin{itemize}
        \item In assenza di crittografia robusta:
        \begin{enumerate}
            \item Intercettare segnali legittimi.
            \item Usare un dispositivo specializzato (es. \texttt{Proxmark}) per imitare i segnali.
        \end{enumerate}
    \end{itemize}
\end{enumerate}

\section{IEEE 802.11 (Wi-Fi)}
Standard per WLAN, interoperabile con Ethernet (IEEE 802.3).

\subsection{Livello Fisico e Accesso al Canale}
\begin{itemize}
    \item Vari \textbf{sotto-standard} (es. 802.11g - 2.4GHz, 802.11ac - 5GHz).
    \item \textbf{Protezioni fisiche:} Si possono usare pitture schermanti.
    \item \textbf{Accesso al Canale:} CSMA/CA (Carrier Sense Multiple Access with Collision Avoidance).
    \item \textbf{Problema del Terminale Nascosto:} Due client (A e C) potrebbero non "sentirsi", ma entrambi "sentono" l'AP (B). Collisione su B se A e C trasmettono contemporaneamente.
\end{itemize}
\begin{figure}[H]
    \centering
    \begin{tikzpicture}[every node/.style={text=lighttext}]
        \node[node_dark] (A) {A};
        \node[node_dark, right=3cm of A] (B) {B (AP)};
        \node[node_dark, right=3cm of B] (C) {C};

        % Ranges
        \begin{scope}[on background layer]
            \fill[blue!50!darkbg, opacity=0.5] (A) circle (2cm);
            \fill[yellow!50!darkbg, opacity=0.5] (C) circle (2cm);
        \end{scope}
        
        \draw[line_dark, <->] (A) -- (B);
        \draw[line_dark, <->] (B) -- (C);
        % A and C cannot hear each other directly but can hear B
        
        \node[label_dark, below=0.2cm of A] {Client A};
        \node[label_dark, below=0.2cm of C] {Client C};
        \node[label_dark, above=0.2cm of B] {Access Point};
        \node[label_dark, text width=3cm, align=center, above left=0.1cm and -1cm of B] {A è un terminale nascosto per C (e viceversa)};
    \end{tikzpicture}
    \caption{Problema del Terminale Nascosto.}
    \label{fig:hidden_terminal}
\end{figure}

\begin{itemize}
    \item \textbf{Inter-Frame Space (IFS):} "Spazi di silenzio" per evitare collisioni.
    Se collisione $\rightarrow$ attesa random (DIFS + backoff).
    \item \textbf{Attacchi legati all'IFS:}
    \begin{itemize}
        \item \textbf{Appropriazione della rete:} Attaccante non rispetta IFS $\rightarrow$ monopolizza canale.
        \item \textbf{Denial of Service:} Due attaccanti non rispettano IFS $\rightarrow$ continue collisioni.
    \end{itemize}
\end{itemize}
(La slide 33 mostra una timeline dell'IFS dove un Attacker trasmette senza rispettare i tempi di attesa (DIFS) dei client legittimi, causando potenzialmente collisioni o guadagnando accesso prioritario).

\subsection{"Misure di Sicurezza" Deboli (Security by Obscurity)}
\begin{enumerate}
    \item \textbf{SSID Hiding:} Non trasmettere l'SSID nei beacon.
    \begin{itemize}
        \item \textbf{Fallacia:} L'SSID è trasmesso in chiaro dai client. "Security by obscurity".
    \end{itemize}
    \item \textbf{MAC Whitelisting:} Permettere solo MAC specifici.
    \begin{itemize}
        \item \textbf{Fallacia:} I MAC sono trasmessi in chiaro e possono essere spoofati.
        L'esempio mostra il comando \texttt{macchanger}:
\begin{minted}{bash}
[root@snark ~]# macchanger -m 11:22:33:44:55:66 virbr0
Current MAC:   52:54:00:b7:9a:84 (unknown)
Permanent MAC: 00:00:00:00:00:00 (XEROX CORPORATION)
\end{minted}
    \end{itemize}
\end{enumerate}

\subsection{Messaggi di Gestione in Chiaro e Attacchi di Deautenticazione/Dissociazione}
\begin{itemize}
    \item Alcuni messaggi di gestione IEEE 802.11 sono inviati \textbf{in chiaro} e \textbf{senza autenticazione}.
    \item Es. messaggio di \textbf{"deautenticazione"} o \textbf{"dissociazione"}.
    \item \textbf{Attacco:} Un attaccante può forgiare un messaggio di deautenticazione, costringendo un client a disconnettersi. Causa disservizio.
\end{itemize}
\begin{figure}[H]
    \centering
    \begin{tikzpicture}[node distance=0.7cm, every node/.style={text=lighttext}]
        % Swimlanes
        \node (client_label) [label_dark, above] {Client};
        \draw[line_dark] (client_label.south west) ++(0,-0.2) -- ++(0,-5) node(client_line_end){};
        \node (ap_label) [label_dark, right=3cm of client_label, above] {Access Point};
        \draw[line_dark] (ap_label.south west) ++(0,-0.2) -- ++(0,-5) node(ap_line_end){};

        % Authentication Phase
        \node[label_dark, left=0.2cm of client_label, yshift=-0.8cm, text width=2cm, align=right] {Authentication phase};
        \draw[arrow_dark] ($(client_label.south)+(0,-1cm)$) -- node[above,sloped] {Request} ($(ap_label.south)+(0,-1cm)$);
        \draw[arrow_dark] ($(ap_label.south)+(0,-1.5cm)$) -- node[above,sloped] {Response} ($(client_label.south)+(0,-1.5cm)$);
        
        % Association Phase
        \node[label_dark, left=0.2cm of client_label, yshift=-2.3cm, text width=2cm, align=right] {Association phase};
        \draw[arrow_dark] ($(client_label.south)+(0,-2.5cm)$) -- node[above,sloped] {Request} ($(ap_label.south)+(0,-2.5cm)$);
        \draw[arrow_dark] ($(ap_label.south)+(0,-3cm)$) -- node[above,sloped] {Response} ($(client_label.south)+(0,-3cm)$);

        % Normal Communication
        \node[label_dark, left=0.2cm of client_label, yshift=-3.8cm, text width=2cm, align=right] {Normal communication};
        \draw[arrow_dark, <->, dashed] ($(client_label.south)+(0,-4cm)$) -- ($(ap_label.south)+(0,-4cm)$);

        % Deauth attack (example injection)
        \node[block_dark, fill=red!70!darkbg, left=0.5cm of client_line_end, yshift=2cm, text width=1.5cm] (attacker) {Attacker};
        \draw[arrow_dark, red, thick] (attacker.east) ++(0,0.2) -- node[above,sloped,black] {Fake Deauth to Client} ($(client_label.south)+(0,-1.8cm)$);
        \draw[arrow_dark, red, thick] (attacker.east) ++(0,-0.2) -- node[below,sloped,black] {Fake Deauth to AP} ($(ap_label.south)+(0,-1.8cm)$);
    \end{tikzpicture}
    \caption{Attacco di Deautenticazione/Dissociazione (schema semplificato di interazione).}
    \label{fig:deauth_attack}
\end{figure}

\subsection{Wired Equivalent Privacy (WEP)}
\begin{itemize}
    \item Primo standard di sicurezza per 802.11. \textbf{ESTREMAMENTE INSICURO!}
    \item Due modalità:
    \begin{enumerate}
        \item \textbf{Shared Key:} Autenticazione challenge-response con chiave condivisa.
        \begin{itemize}
            \item \textbf{Vulnerabilità:} Attacchi KPA per ricavare la chiave.
        \end{itemize}
        \item \textbf{Open System:} Nessuna vera autenticazione della chiave.
    \end{enumerate}
    \item \textbf{Cifratura WEP:}
    \begin{itemize}
        \item Utilizza \textbf{RC4}.
        \item Chiave base $k$ + Vettore di Inizializzazione (IV) $v$ (24 bit, in chiaro) $\rightarrow$ keystream RC4($v,k$).
        \item \textbf{Vulnerabilità dell'IV:}
        \begin{itemize}
            \item \textbf{Collisioni IV:} IV di soli 24 bit $\rightarrow$ riutilizzo frequente (birthday paradox).
            $P(\text{collisione}) \approx 1 - e^{-N^2 / (2 \cdot 2^{24})}$. Con $N \approx 30000$, $P \approx 1$.
            \item Riutilizzo IV con stessa chiave in RC4 è catastrofico: $C_1 = P_1 \oplus K_s$, $C_2 = P_2 \oplus K_s \Rightarrow C_1 \oplus C_2 = P_1 \oplus P_2$.
            \item Permette attacchi statistici e di replay.
        \end{itemize}
    \end{itemize}
\end{itemize}

\begin{figure}[H]
\centering
\begin{tikzpicture}[node distance=0.7cm, every node/.style={text=lighttext}]
    \node (client_label) [label_dark, above] {Client Station};
    \draw[line_dark] (client_label.south west) ++(0,-0.2) -- ++(0,-3.5);
    \node (ap_label) [label_dark, right=3cm of client_label, above] {Access Point};
    \draw[line_dark] (ap_label.south west) ++(0,-0.2) -- ++(0,-3.5);

    \draw[arrow_dark] ($(client_label.south)+(0,-0.5cm)$) -- node[above,sloped] {Auth request} ($(ap_label.south)+(0,-0.5cm)$);
    \draw[arrow_dark] ($(ap_label.south)+(0,-1.2cm)$) -- node[above,sloped] {$n$ = challenge} ($(client_label.south)+(0,-1.2cm)$);
    \draw[arrow_dark] ($(client_label.south)+(0,-1.9cm)$) -- node[above,sloped] {RC4$_{wep\_key}(n)$} ($(ap_label.south)+(0,-1.9cm)$);
    \draw[arrow_dark, dashed] ($(ap_label.south)+(0,-2.6cm)$) -- node[above,sloped] {Accept/Reject} ($(client_label.south)+(0,-2.6cm)$);
\end{tikzpicture}
\caption{Autenticazione WEP Shared Key.}
\label{fig:wep_shared_key}
\end{figure}

\begin{figure}[H]
    \centering
    \begin{tikzpicture}[every node/.style={text=lighttext}]
        \draw[line_dark] (0,0) rectangle (5,0.5) node[midway] {IV (24 bit)};
        \node[label_dark, right=0.1cm of {(5,0.25)}] {Key ID (opz.)};
        \draw[line_dark] (0,-0.2) rectangle (5,-0.7) node[midway] {Data};
        \draw[line_dark] (0,-1.4) rectangle (5,-1.9) node[midway] {ICV (CRC32)};
        
        % Braces for encrypted part
        \draw [decorate,decoration={brace,amplitude=5pt,mirror}, line_dark]
              (5.2,0.5) -- (5.2,-0.7) node [label_dark,midway,xshift=0.8cm] {Encrypted};
        
        \node[label_dark, below=0.5cm of {(2.5,-1.9)}] {Pacchetto WEP (semplificato)};
    \end{tikzpicture}
    \caption{Struttura Frame WEP (semplificata).}
    \label{fig:wep_frame}
\end{figure}


\subsection{Wi-Fi Protected Access (WPA, WPA2, WPA3) e IEEE 802.11i}
Sviluppati per ovviare ai problemi di WEP. IEEE 802.11i è lo standard.
\begin{itemize}
    \item \textbf{Modalità di utilizzo:}
    \begin{itemize}
        \item \textbf{PSK (Pre-Shared Key) / WPA-Personal:} Chiave segreta (passphrase) condivisa.
        \item \textbf{Enterprise / WPA-Enterprise (802.1X):} Autenticazione individuale tramite server RADIUS.
        \item \textbf{WPS (Wi-Fi Protected Setup):} Autenticazione facilitata (vulnerabile).
    \end{itemize}
    \item \textbf{Protocolli di Sicurezza IEEE 802.11i:}
    \begin{enumerate}
        \item \textbf{TKIP (Temporal Key Integrity Protocol):}
        \begin{itemize}
            \item "Pezza" per WEP. \textbf{DEPRECATO E INSICURO.}
            \item Integrità: MIC "Michael" (64 bit).
            \item Confidenzialità: RC4 (con key mixing).
            \item Sequence Counter (TSC) per anti-replay e chiave dinamica.
        \end{itemize}
        \item \textbf{CCMP (Counter Mode with CBC MAC Protocol):}
        \begin{itemize}
            \item Basato su \textbf{AES}. \textbf{RACCOMANDATO E SICURO} (WPA2/WPA3).
            \item Integrità: AES-CBC-MAC.
            \item Confidenzialità: AES-CTR.
            \item Contatore di pacchetto (48 bit) per nonce anti-replay.
        \end{itemize}
    \end{enumerate}
\end{itemize}

\begin{figure}[H]
    \centering
    \begin{tikzpicture}[every node/.style={text=lighttext, font=\footnotesize}]
        \coordinate (start) at (0,0);
        \node[draw, minimum height=1cm, minimum width=1.5cm, anchor=west] at (start) (mac_ctrl) {MAC Control};
        \node[draw, minimum height=1cm, minimum width=2.5cm, anchor=west, right=0 of mac_ctrl.east] (dest_addr) {Destination MAC Address};
        \node[draw, minimum height=1cm, minimum width=2.5cm, anchor=west, right=0 of dest_addr.east] (src_addr) {Source MAC Address};
        \node[draw, minimum height=1cm, minimum width=4cm, anchor=west, right=0 of src_addr.east] (msdu) {MAC Service Data Unit (MSDU)};
        \node[draw, minimum height=1cm, minimum width=1.5cm, anchor=west, right=0 of msdu.east] (crc) {CRC};
        
        % MAC Header Brace
        \draw [decorate,decoration={brace,amplitude=5pt}, line_dark]
              ($(mac_ctrl.south west)+(-0.1,-0.1)$) -- ($(src_addr.south east)+(0.1,-0.1)$) 
              node [label_dark,midway,yshift=-0.5cm] {MAC header};

        % MAC Trailer Brace
        \draw [decorate,decoration={brace,amplitude=5pt}, line_dark]
              ($(crc.south west)+(-0.1,-0.1)$) -- ($(crc.south east)+(0.1,-0.1)$) 
              node [label_dark,midway,yshift=-0.5cm] {MAC trailer};
    \end{tikzpicture}
    \caption{Formato Generale IEEE 802 MPDU.}
    \label{fig:mpdu_format}
\end{figure}

\begin{itemize}
    \item \textbf{Attacchi a WPA-PSK:}
    \begin{itemize}
        \item Vulnerabile ad attacchi \textbf{brute-force} o a \textbf{dizionario} sulla passphrase se debole (catturando l'handshake a 4 vie).
    \end{itemize}
\end{itemize}

\subsection{WPS (Wi-Fi Protected Setup)}
\begin{itemize}
    \item Metodo per semplificare la connessione (tasto fisico, PIN).
    \item \textbf{Vulnerabilità di WPS (PIN):}
    \begin{itemize}
        \item PIN di 8 cifre (7 significative + checksum).
        \item \textbf{Errore di progettazione:} Verificato in due metà (4 cifre, poi 3).
        \item Spazio di brute force ridotto a $10^4 + 10^3 = 11.000$ tentativi.
        \item Attacco brute-force recupera PIN in poche ore $\rightarrow$ ottiene chiave WPA/WPA2.
        \item \textbf{Pixiedust Attack:} PRNG predicibile in alcuni dispositivi $\rightarrow$ attacco in minuti.
    \end{itemize}
\end{itemize}

\subsection{Rogue AP (Access Point Falso/Non Autorizzato)}
\begin{itemize}
    \item AP installato senza autorizzazione.
    \item Può \textbf{spoofare SSID} di reti legittime per intercettare traffico o credenziali.
\end{itemize}

\subsection{KRACK (Key Reinstallation Attacks)}
\begin{itemize}
    \item Vulnerabilità in WPA2 (nell'handshake, non in AES).
    \item Permette a un attaccante di forzare la \textbf{reinstallazione di una chiave di sessione già utilizzata}.
    \item Può portare a decifratura parziale del traffico, iniezione, ecc.
    \item Ha contribuito allo sviluppo di \textbf{WPA3} (che include protezioni).
\end{itemize}

\end{document}