\input{preambolo_comune}

\title{Laboratorio 3 (parte 1 e 2)}
\author{Alessandro Amella, Gemini e Claude}
\date{\today}

\begin{document}

\maketitle
\tableofcontents
\newpage

\section{Laboratorio \#3: Applied Cryptography}

\subsection{Concetti Fondamentali}

\subsubsection{AES (Advanced Encryption Standard)}
L'AES è un cifrario simmetrico a blocchi ampiamente utilizzato.
\begin{itemize}
    \item \textbf{Dimensione Blocco:} 128 bit.
    \item \textbf{Lunghezze Chiave:} 128, 192, o 256 bit.
\end{itemize}

\subsubsection*{Modalità Operative AES}
Le modalità operative definiscono come i blocchi di dati vengono processati.
\begin{itemize}
    \item \textbf{ECB (Electronic CodeBook):}
    \begin{itemize}
        \item La modalità più semplice.
        \item Ogni blocco di plaintext viene cifrato indipendentemente con la stessa chiave.
        \item \textbf{Debolezza:} Blocchi di plaintext identici producono blocchi di ciphertext identici. Questo non nasconde i pattern dei dati, rendendolo insicuro per molte applicazioni (come dimostrato dall'esperimento Tux).
    \end{itemize}
    \item \textbf{CBC (Cipher Block Chaining):}
    \begin{itemize}
        \item Più sicura di ECB.
        \item Utilizza un \textbf{IV (Initialization Vector)} casuale per il primo blocco.
        \item Ogni blocco di plaintext viene sottoposto a XOR con il blocco di ciphertext precedente prima di essere cifrato. Questo crea una dipendenza ("chaining") tra i blocchi.
    \end{itemize}
\end{itemize}

\subsubsection{OpenSSL}
OpenSSL è una potente libreria e strumento da riga di comando per la crittografia e la gestione di certificati.
\begin{itemize}
    \item \textbf{Cifratura/Decifratura AES:}
    \begin{minted}[fontsize=\small, baselinestretch=1.0, frame=lines, framesep=2mm, breaklines, tabsize=4]{bash}
# Cifratura AES-256-ECB
openssl aes-256-ecb -e -in plaintext.txt -out ciphertext.enc

# Decifratura AES-256-ECB
openssl aes-256-ecb -d -in ciphertext.enc -out decrypted.txt

# Cifratura AES-256-CBC (con IV specificato)
openssl aes-256-cbc -e -iv <IV_IN_HEX> -in plaintext.txt -out ciphertext.enc

# Decifratura AES-256-CBC (con IV specificato)
openssl aes-256-cbc -d -iv <IV_IN_HEX> -in ciphertext.enc -out decrypted.txt
    \end{minted}
    \item \textbf{PBKDF (Password-Based Key Derivation Function):}
    OpenSSL usa la password fornita dall'utente per derivare una chiave crittografica della lunghezza richiesta (e un IV, se necessario e non fornito).
    \begin{itemize}
        \item \textbf{Salt:} Un valore casuale aggiunto alla password prima della derivazione della chiave. Previene attacchi basati su tabelle precalcolate (rainbow tables) per la stessa password e rende unica la chiave derivata anche se la password è la stessa usata altrove. Se non si usa \texttt{-nosalt}, OpenSSL genera un salt e lo prepone al ciphertext (con il prefisso "Salted\_\_").
        \item Opzioni utili:
        \begin{itemize}
            \item \texttt{-e}: encrypt
            \item \texttt{-d}: decrypt
            \item \texttt{-p}: stampa la chiave derivata, il salt e l'IV (se usato)
            \item \texttt{-nosalt}: disabilita l'uso del salt (sconsigliato)
            \item \texttt{-pbkdf2}: usa l'algoritmo PBKDF2 (più robusto del default)
            \item \texttt{-iter N}: specifica il numero di iterazioni per PBKDF2
            \item \texttt{-iv HEX\_IV}: specifica l'IV in esadecimale (per CBC, OFB, CFB)
        \end{itemize}
    \end{itemize}
    \item \textbf{Dimensione File Cifrati:}
    Solitamente i file cifrati sono leggermente più grandi del plaintext a causa di:
    \begin{itemize}
        \item \textbf{Salt:} Se usato e memorizzato nell'header del file cifrato (tipicamente 8 byte per il salt + 8 byte per "Salted\_\_").
        \item \textbf{Padding:} Necessario per assicurare che il plaintext sia un multiplo della dimensione del blocco (es. 128 bit per AES). Questo è rilevante per modalità come ECB e CBC.
    \end{itemize}
\end{itemize}

\subsubsection{Certificati TLS/SSL \& Let's Encrypt}
I certificati TLS/SSL sono usati per autenticare i server e stabilire connessioni sicure (HTTPS).
\begin{itemize}
    \item \textbf{Let's Encrypt:} Una Certificate Authority (CA) gratuita, automatizzata e aperta.
    \item \textbf{Verifica con OpenSSL:}
    \begin{minted}[fontsize=\small, frame=lines, framesep=2mm, breaklines, tabsize=4]{bash}
# Visualizza informazioni certificato (issuer, subject, date, ecc.)
openssl s_client -connect nomesito.com:443 2>/dev/null | openssl x509 -noout -text

# Visualizza dettagli connessione TLS (protocollo, cipher suite)
openssl s_client -connect nomesito.com:443 2>/dev/null

# Testa supporto per una specifica versione TLS (es. TLS 1.2)
openssl s_client -connect nomesito.com:443 -tls1_2
    \end{minted}
    \item \textbf{Analisi con SSL Labs:} Il sito \url{https://www.ssllabs.com/ssltest/} fornisce un'analisi dettagliata della configurazione SSL/TLS di un server.
\end{itemize}

\subsubsection{PGP (Pretty Good Privacy) / GPG (GNU Privacy Guard)}
GPG è l'implementazione open-source di OpenPGP.
\begin{itemize}
    \item Utilizza crittografia ibrida: asimmetrica per scambiare una chiave di sessione simmetrica, che viene poi usata per cifrare i dati.
    \item Utile per cifrare file, email e verificare firme digitali.
    \item Si basa su un modello di fiducia chiamato "Web of Trust" (alternativo al modello gerarchico delle CA).
\end{itemize}

\subsection{Esercizio 1: Check Certificates}
\subsubsection{Obiettivo}
Verificare i certificati SSL/TLS di siti web.

\subsubsection{Procedura}
\begin{enumerate}
    \item \textbf{Trovare siti con/senza Let's Encrypt:}
    \begin{itemize}
        \item Usare il comando:
        \begin{minted}[fontsize=\small, frame=lines, framesep=2mm, breaklines, tabsize=4]{bash}
openssl s_client -connect nomesito.com:443 2>/dev/null | openssl x509 -noout -text -issuer
        \end{minted}
        \item Cercare "Let's Encrypt" nel campo \texttt{Issuer}.
    \end{itemize}
    \item \textbf{Verificare versioni TLS supportate:}
    \begin{itemize}
        \item Per ogni versione TLS (es. \texttt{-tls1}, \texttt{-tls1\_1}, \texttt{-tls1\_2}, \texttt{-tls1\_3}):
        \begin{minted}[fontsize=\small, frame=lines, framesep=2mm, breaklines, tabsize=4]{bash}
openssl s_client -connect nomesito.com:443 -<versione_tls>
        \end{minted}
        Se la connessione ha successo, la versione è supportata.
        \item Oppure, analizzare l'output di:
        \begin{minted}[fontsize=\small, frame=lines, framesep=2mm, breaklines, tabsize=4]{bash}
openssl s_client -connect nomesito.com:443 2>/dev/null
        \end{minted}
        cercando la riga "Protocol : TLSvX.Y".
    \end{itemize}
    \item \textbf{Analizzare con SSL Labs:}
    \begin{itemize}
        \item Visitare \url{https://www.ssllabs.com/ssltest/}.
        \item Inserire l'hostname del sito da analizzare.
        \item Commentare il rating generale, il supporto dei protocolli, la robustezza delle cipher suite, ecc.
    \end{itemize}
\end{enumerate}

\subsection{Esercizio 2: Decrypt the Files}
\subsubsection{Obiettivo Generale}
Trovare le password originali partendo dagli hash forniti, poi usare le password per decifrare i file e trovare le flag.

\subsubsection{Craccare gli Hash}
\begin{enumerate}
    \item \textbf{Identificare il tipo di Hash:} Basarsi sulla lunghezza dell'hash in caratteri esadecimali.
    \begin{itemize}
        \item MD5: 32 caratteri
        \item SHA1: 40 caratteri
        \item SHA256: 64 caratteri
        \item SHA512: 128 caratteri
    \end{itemize}
    \item \textbf{Strumenti:}
    \begin{itemize}
        \item \textbf{Online:} CrackStation, MD5Online (utili per hash semplici o comuni).
        \item \textbf{Offline:} John the Ripper (JtR), Hashcat. Richiedono una wordlist (es. \texttt{rockyou.txt}).
        \begin{minted}[fontsize=\small, frame=lines, framesep=2mm, breaklines, tabsize=4]{bash}
# Esempio con John the Ripper (l'hash deve essere in un file)
# john --wordlist=/path/to/rockyou.txt --format=raw-sha1 hash_file.txt
        \end{minted}
    \end{itemize}
\end{enumerate}

\subsubsection{File 1: \texttt{file1.aes-256-ecb.txt}}
\begin{itemize}
    \item \textbf{Algoritmo:} AES-256-ECB.
    \item \textbf{Hash Password 1 (SHA1):} \texttt{9e02fdbbc6df842939b75a98e92e98317d29046c}
    \item \textbf{Procedura:}
    \begin{enumerate}
        \item Craccare l'hash SHA1 per ottenere \texttt{PASSWORD\_1}.
        \item Decifrare con OpenSSL:
        \begin{minted}[fontsize=\small, frame=lines, framesep=2mm, breaklines, tabsize=4]{bash}
openssl aes-256-ecb -d -in file1.aes-256-ecb.txt -out file1.decrypted.txt
# Verrà richiesta PASSWORD_1
        \end{minted}
        \item Ispezionare \texttt{file1.decrypted.txt} per la flag.
    \end{enumerate}
\end{itemize}

\subsubsection{File 2: \texttt{file2.aes-256-cbc.txt}}
\begin{itemize}
    \item \textbf{Algoritmo:} AES-256-CBC.
    \item \textbf{Hash Password 2a (MD5):} \texttt{977cd0822fe6f0a1ca64787f933820f3} (suggerimento: semplice, da wordlist diretta).
    \item \textbf{Hash Password 2b (SHA512 con salt):} \texttt{3b2fd659c8f48db0688148fabe3cd9d37e928e9b...}
    L'hash è stato calcolato su \texttt{PASSWORD\_2B + "kjsahdjhasd"}. Questo "salt" esterno deve essere considerato nel processo di cracking (non come parametro OpenSSL).
    \item \textbf{Procedura:}
    \begin{enumerate}
        \item Craccare l'MD5 per ottenere \texttt{PASSWORD\_2A}.
        \item Craccare lo SHA512 (considerando il salt \texttt{kjsahdjhasd} concatenato alla password prima dell'hashing) per ottenere \texttt{PASSWORD\_2B}.
        \item Decifrare con OpenSSL (OpenSSL di solito deriva l'IV dalla password e dal salt interno al file, se presente):
        \begin{minted}[fontsize=\small, frame=lines, framesep=2mm, breaklines, tabsize=4]{bash}
# Prova con PASSWORD_2A
openssl aes-256-cbc -d -in file2.aes-256-cbc.txt -out file2a.decrypted.txt
# Prova con PASSWORD_2B
openssl aes-256-cbc -d -in file2.aes-256-cbc.txt -out file2b.decrypted.txt
# Per debug, usare -p per vedere salt, chiave e IV usati da OpenSSL:
# openssl aes-256-cbc -d -p -in file2.aes-256-cbc.txt
        \end{minted}
        \item Ispezionare i file decifrati per la flag.
    \end{enumerate}
\end{itemize}

\subsubsection{File 3: \texttt{file3.enc.txt}}
\begin{itemize}
    \item \textbf{Algoritmo presunto:} GPG (data la natura degli esercizi).
    \item \textbf{Hash Password 3 (SHA256):} \texttt{e547671ff6e7e1d7b5fc2141f6b63648b45a2df2d1a7270ed9f0773a79109007}
    \item \textbf{Procedura:}
    \begin{enumerate}
        \item Craccare l'hash SHA256 per ottenere \texttt{PASSWORD\_3}.
        \item Decifrare con GPG:
        \begin{minted}[fontsize=\small, frame=lines, framesep=2mm, breaklines, tabsize=4]{bash}
# Modo batch (non interattivo, la password è fornita)
gpg --pinentry-mode loopback --batch --passphrase "PASSWORD_3" \
    -o file3.decrypted.txt file3.enc.txt

# Modo interattivo (GPG chiederà la password)
# gpg -d -o file3.decrypted.txt file3.enc.txt
        \end{minted}
        \item Ispezionare \texttt{file3.decrypted.txt} per la flag.
    \end{enumerate}
\end{itemize}

\subsection{Esperimento Tux: Dimostrazione Debolezza ECB}
\subsubsection{Obiettivo}
Dimostrare visivamente che la modalità ECB non nasconde i pattern dei dati.
\subsubsection{File}
\texttt{tux.ppm} (immagine Portable PixMap).

\subsubsection{Procedura}
\begin{enumerate}
    \item \textbf{Installare GIMP (se necessario):}
    \begin{minted}[fontsize=\small, frame=lines, framesep=2mm, breaklines, tabsize=4]{bash}
sudo apt update && sudo apt install gimp
    \end{minted}
    \item \textbf{Ispezionare il formato PPM (opzionale):}
    Un file PPM (formato P6) ha un header testuale (3 righe: magic number P6, larghezza altezza, valore massimo colore) seguito dai dati raw dei pixel.
    \begin{minted}[fontsize=\small, frame=lines, framesep=2mm, breaklines, tabsize=4]{bash}
xxd -g 3 -c 15 tux.ppm | head
    \end{minted}
    \item \textbf{Separare header e body:}
    \begin{minted}[fontsize=\small, frame=lines, framesep=2mm, breaklines, tabsize=4]{bash}
head -n 3 tux.ppm > tux.header
tail -n +4 tux.ppm > tux.body
    \end{minted}
    \item \textbf{Cifrare il body in modalità ECB:}
    Usare una password semplice, es. \texttt{"testpassword"}.
    \begin{minted}[fontsize=\small, frame=lines, framesep=2mm, breaklines, tabsize=4]{bash}
openssl aes-256-ecb -e -in tux.body -out tux.body.ecb
# Verrà richiesta la password
    \end{minted}
    \item \textbf{Riassemblare l'immagine cifrata:}
    \begin{minted}[fontsize=\small, frame=lines, framesep=2mm, breaklines, tabsize=4]{bash}
cat tux.header tux.body.ecb > tux.ecb.ppm
    \end{minted}
    \item \textbf{Visualizzare l'immagine cifrata \texttt{tux.ecb.ppm}:}
    Aprire con GIMP. Si noterà che la sagoma di Tux è ancora riconoscibile.
    \item \textbf{Opzionale: Prova con CBC:}
    \begin{minted}[fontsize=\small, frame=lines, framesep=2mm, breaklines, tabsize=4]{bash}
# Genera un IV casuale di 16 byte (32 caratteri hex)
IV_HEX=$(openssl rand -hex 16)
echo "Using IV: $IV_HEX"

# Cifra con CBC usando la stessa password e l'IV generato
openssl aes-256-cbc -e -iv "$IV_HEX" -in tux.body -out tux.body.cbc
# Verrà richiesta la password (es. "testpassword")

cat tux.header tux.body.cbc > tux.cbc.ppm
    \end{minted}
    Aprire \texttt{tux.cbc.ppm} con GIMP. L'immagine dovrebbe apparire come rumore casuale, dimostrando l'efficacia di CBC nel nascondere i pattern.
\end{enumerate}

\newpage
\section{Laboratorio \#3A: Wireshark \& Network Attacks}

\subsection{Concetti Fondamentali}

\subsubsection{Wireshark}
Wireshark è un analizzatore di protocolli di rete.
\begin{itemize}
    \item \textbf{Analisi:} Real-time o offline (da file \texttt{.pcap}, \texttt{.pcapng}).
    \item \textbf{Interfaccia:} Mostra una lista di pacchetti catturati e permette di ispezionare i dettagli di ogni layer protocollare.
    \item \textbf{Filtri (Display Filters):} Potenti per isolare pacchetti specifici. Esempi:
    \begin{itemize}
        \item \texttt{http}
        \item \texttt{tcp.port == 80}
        \item \texttt{ip.addr == 192.168.1.1}
        \item \texttt{http.request.method == "GET"}
        \item \texttt{tls.handshake.extension.type == "server\_name"} (per Server Name Indication)
    \end{itemize}
    \item \textbf{Follow Stream:} Permette di ricostruire e visualizzare la conversazione completa per protocolli come TCP, HTTP, TLS.
\end{itemize}

\subsubsection{Sicurezza Reti WiFi}
\begin{itemize}
    \item \textbf{WEP (Wired Equivalent Privacy):} Obsoleto e insicuro. Usa RC4 con IV corti (24 bit) che possono essere facilmente craccati.
    \item \textbf{WPA/WPA2/WPA3 (Wi-Fi Protected Access):}
    \begin{itemize}
        \item \textbf{WPA/WPA2:} Usano TKIP (WPA) o AES-CCMP (WPA2). Vulnerabili ad attacchi offline sulla password (PSK - Pre-Shared Key) se questa è debole. L'attacco richiede la cattura del \textbf{4-way handshake} che avviene quando un client si connette all'AP.
        \item \textbf{WPA3:} Introduce maggiore sicurezza, come la protezione contro attacchi di dizionario offline anche con password deboli (grazie a SAE - Simultaneous Authentication of Equals).
    \end{itemize}
\end{itemize}

\subsubsection{Monitor Mode (Modalità Monitor)}
Una modalità operativa delle schede di rete wireless che permette di catturare tutti i pacchetti WiFi trasmessi su un determinato canale, indipendentemente dal fatto che siano indirizzati o meno al MAC address della scheda.
\begin{itemize}
    \item Essenziale per lo sniffing del traffico WiFi, inclusa la cattura del 4-way handshake.
    \item L'attivazione dipende dalla scheda wireless e dal driver. Strumenti come \texttt{airmon-ng} (parte della suite Aircrack-ng) possono aiutare.
\end{itemize}

\subsubsection{Aircrack-ng Suite}
Una collezione di strumenti per l'auditing della sicurezza delle reti wireless.
\begin{itemize}
    \item \texttt{airmon-ng}: Per gestire la modalità monitor.
    \item \texttt{airodump-ng}: Per catturare pacchetti e scoprire reti.
    \item \texttt{aireplay-ng}: Per iniettare pacchetti (es. deautenticazione per forzare un handshake).
    \item \texttt{aircrack-ng}: Per craccare chiavi WEP e password WPA/WPA2-PSK da file di cattura (\texttt{.cap}) contenenti il 4-way handshake, usando una wordlist.
\end{itemize}

\subsubsection{ARP Poisoning (ARP Spoofing)}
Un attacco Man-in-the-Middle (MitM) su reti locali (LAN).
\begin{itemize}
    \item L'attaccante invia messaggi ARP falsificati per associare il proprio MAC address all'indirizzo IP della vittima (o del gateway).
    \item Tutto il traffico della vittima destinato al gateway (e viceversa) passa attraverso l'attaccante, che può intercettarlo o modificarlo.
\end{itemize}

\subsection{Esercizio: Crack WiFi passwd}
\subsubsection{Obiettivo}
Trovare la password di una rete WiFi WPA/WPA2 analizzando un file di cattura del traffico (\texttt{WIFI-analisi1.cap}).

\subsubsection{Procedura}
\begin{enumerate}
    \item \textbf{Preparare una Wordlist:}
    Assicurarsi di avere una wordlist, es. \texttt{rockyou.txt}. Se è compressa (\texttt{.gz}), decomprimerla.
    \begin{minted}[fontsize=\small, frame=lines, framesep=2mm, breaklines, tabsize=4]{bash}
# Esempio (il percorso potrebbe variare):
# sudo gzip -d /usr/share/wordlists/rockyou.txt.gz
    \end{minted}
    \item \textbf{Opzionale: Ispezionare il file \texttt{.cap} con Wireshark:}
    \begin{itemize}
        \item Aprire \texttt{WIFI-analisi1.cap} in Wireshark.
        \item Cercare i pacchetti del 4-way handshake (filtrando per \texttt{eapol}). Wireshark solitamente li identifica.
        \item Identificare il BSSID (MAC address dell'Access Point) e l'ESSID (nome della rete) dai Beacon frames o Probe Responses.
    \end{itemize}
    \item \textbf{Usare \texttt{aircrack-ng}:}
    \begin{minted}[fontsize=\small, frame=lines, framesep=2mm, breaklines, tabsize=4]{bash}
aircrack-ng -w /percorso/della/tua/wordlist.txt -b <BSSID_DELL_AP> WIFI-analisi1.cap
    \end{minted}
    Sostituire \texttt{<BSSID\_DELL\_AP>} con il MAC address dell'AP target. Se c'è una sola rete con handshake nel file, \texttt{aircrack-ng} potrebbe individuarla automaticamente e l'opzione \texttt{-b} potrebbe non essere strettamente necessaria (ma è buona prassi specificarla).
    \item Se la password è presente nella wordlist, \texttt{aircrack-ng} la mostrerà come "KEY FOUND! [password]".
\end{enumerate}

\subsection{Esercizio: Detect the intruder}
\subsubsection{Obiettivo}
Analizzare il traffico nel file \texttt{s3cret.pcapng} per trovare una flag nel formato \texttt{CCIT\{flag\}}. L'indizio è che un "intruder" ha visitato una pagina web con estensione HTML.

\subsubsection{Procedura}
\begin{enumerate}
    \item \textbf{Aprire \texttt{s3cret.pcapng} con Wireshark.}
    \item \textbf{Filtrare per traffico HTTP:}
    Applicare il display filter: \texttt{http}.
    \item \textbf{Cercare richieste a pagine HTML:}
    Si può raffinare il filtro:
    \begin{minted}[fontsize=\small, frame=lines, framesep=2mm, breaklines, tabsize=4]{text}
http.request.method == "GET" and http.request.uri contains ".html"
    \end{minted}
    \item \textbf{Ispezionare i pacchetti HTTP risultanti:}
    \begin{itemize}
        \item Controllare le richieste GET e le relative risposte (codice 200 OK).
        \item La flag potrebbe trovarsi nell'URI, in un header HTTP (User-Agent, Cookie, header custom), o nel corpo della risposta HTML.
    \end{itemize}
    \item \textbf{Usare "Follow HTTP Stream":}
    Fare click destro su un pacchetto HTTP sospetto (idealmente la risposta che contiene la pagina HTML) e selezionare "Follow" $\rightarrow$ "HTTP Stream".
    \item Cercare la stringa \texttt{CCIT\{} all'interno del contenuto dello stream HTTP.
\end{enumerate}

\newpage
\section{Consigli Generali per l'Esame}
\begin{itemize}
    \item \textbf{Familiarità con la CLI:} Esercitarsi con \texttt{openssl}, \texttt{gpg}, \texttt{aircrack-ng}, e comandi base di Linux (\texttt{head}, \texttt{tail}, \texttt{cat}, \texttt{xxd}, \texttt{grep}).
    \item \textbf{Comprensione dei Concetti:} Non imparare solo i comandi a memoria. Capire il \textit{perché} si usa una certa modalità di cifratura, cos'è un IV, un salt, un 4-way handshake, il ruolo dei certificati, ecc.
    \item \textbf{Cracking di Hash e Password:} Sapere come identificare i tipi di hash, usare tool online/offline e dove trovare wordlist comuni (es. in Kali Linux: \texttt{/usr/share/wordlists/}).
    \item \textbf{Wireshark:} Allenarsi con i display filters e la funzione "Follow Stream". È uno strumento fondamentale per l'analisi di rete.
    \item \textbf{Problem Solving:} Leggere attentamente i messaggi di errore. Se un comando non funziona come previsto, consultare le pagine \texttt{man} (\texttt{man <comando>}) o l'opzione \texttt{--help}. Ad esempio, un errore "bad decrypt" in OpenSSL solitamente indica una password errata, un IV errato (se specificato manualmente), o una modalità di cifratura/algoritmo non corretti.
\end{itemize}

\end{document}